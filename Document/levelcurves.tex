\chapter{Cooperative Level Curve Tracking \label{ch:levelcurves}}


In this chapter, two distributed controllers enabling multi-agent systems to track level curves are proposed.  Agents are modelled as double integrators and LTI systems.  This control strategy maintains agents in a desired formation and steers the whole group to the aimed level curve. Agents exchange information with neighbours using only a constrained communication topology. Finally,  convergence of the presented control laws is proved. Results are supported by simulations for double integrator models  and quad-rotor helicopter formations.


%%%%%%%%%%%%%%%%%%%%%%%%%%%%%%%%%%%%%%%%%%%%%%%%%%%%%%
 

\section{Introduction}
As exposed, exploration and mapping of environmental scalar fields using autonomous systems is highly relevant nowadays. Level curve tracking (isolines) of concentration fields such as temperature maps (isotherms), pressure maps (isobars), altitude maps, salinity or concentration of toxic substances are only a few examples of tasks of growing interest. 

The problem of spatial mapping and level curve tracking has been  addressed by \citep{OgrenFiorelliLeonard04, ZhangFiorelliLeonard07, ZhangLeonard10, WilliamsSukhatme12}. However, these approaches require special formation shapes,  as well as information about estimated gradient, estimated Hessian matrix and position of the formation's center of mass.

In this chapter, based on the cooperative source seeking strategy for single, double integrator models and LTI systems presented in Chapters \ref{ch:single_double} and \ref{ch:lti_model}, the problem of tracking a level curve inside of an unknown scalar field using a $N$ identical agents formation under undirected constrained communication is addressed.  
To solve the \textbf{level curve tracking problem} as defined in Chapter \ref{ch:introduction}, simple distributed control laws for agents operating under limited communication will be presented.
If signals are corrupted by noise, distributed consensus filters are used. 
Note that in the solution here implemented, in contrast to previously cited authors, it is not necessary to know or estimate gradient and Hessian matrix at the center of mass. Applying the present approach the center of mass' position needs not to be known.



This chapter is organized as follows. In Section \ref{sec:lc_background},problem statement and definition of closed level curves are introduced. In Sections \ref{sec:lc_doubleint}  and \ref{sec:lc_ltimodel}, distributed control laws and stability analysis for double integrator models and LTI systems are presented. Simulation results illustrate the feasibility of the proposed approach in Section \ref{sec:lc_simul}. Finally, Section \ref{sec:lc_concl} presents conclusions to this chapter.



%%%%%%%%%%%%%%%%%%%%%%%%%%%%%%%%%%%%%%%%%%%%%%%%%%%%%%%%%%%%%%%%%%%%%%%%%%%%


\section{Background and Problem Statement} \label{sec:lc_background}
%In this section, some facts from algebraic graph theory, notation and mathematical preliminaries, assumptions and the problem setup are outlined.


%\subsection{Notation}
%
%Let $\mathcal{G=(V,E)}$ be an undirected graph that models the interaction among agents where $\mathcal{V}=\{1,...,N\}$ is the set of nodes and  $\mathcal{E\subseteq V \times V}$ is the set of edges. Each node represents an agent and each edge corresponds to an information exchange channel. An edge $(i,j) \in  \mathcal{E}$ indicates that the agent $i$ and $j$ exchange information.  Let $\mathcal{N}_i = \{ j \in \mathcal{V}: a_{ij} \not=0 \}$ denote the set of neighbors of node $i$. % and $\mathcal{J}_i = \mathcal{N}_i  \cup \{ i \}$ denote the set of inclusive neighbors of node $i$. 
%The  adjacency matrix $\mathcal{A}=[a_{ij}]  \in \mathbb{R}^{N \times N}$ of a graph $\mathcal{G(V,E)}$ with $N$ nodes specifies  the interconnection topology of the network. Here $a_{ij}=1$ if $(i,j)\in \mathcal{E}$, else  $a_{ij}=0$. Note that here $\mathcal{A}$ is symmetric. 
%The Laplacian matrix $\mathcal{L}$ of the graph $\mathcal{G}$ is defined as $\mathcal{L}=\Delta-\mathcal{A}$, where $\Delta=$diag$(\mathcal{A}\cdot \boldsymbol{1})$ is a diagonal matrix  with the sensors' degrees on its diagonal, i.e.,  $\Delta_{ii}=d_i=\sum_j a_{ij}$. Here  $\boldsymbol{1}=[1\ 1\  1\  ... \ 1]^T   \in \mathbb{R}^{N }$ denotes the vector of ones which is a right eigenvector of $\mathcal{L}$ corresponding to $\lambda_1=0$, i.e., $\mathcal{L} \cdot \boldsymbol{1}=0$. 
%%The Laplacian matrix also satisfies $\mathcal{L}  M= \mathcal{L}$, where 
%A projection matrix is defined as $M = I-\frac{1}{N}\boldsymbol{1}\boldsymbol{1}^T \in \mathbb{R}^{N \times N}$ and $M$ satisfies $M \boldsymbol{1}=0$. The second smallest eigenvalue $\lambda_2$ of $\mathcal{L}$ determines the convergence speed of the algorithm.
%$ \left| \mathcal{N}_i \right| $ indicates the cardinality  of the set of neighbors $\mathcal{N}_i$, and $\otimes$ denotes the Kronecker product.
%$I$ denotes the identity matrix of appropriate dimension.  
%
%In the case of LTI systems the normalized laplacian matrix $\mathcal{L}$ proposed by \citep{FaxMurray04} and \citep{PopovWerner12} is used, where $a_{ij}=\frac{1}{ \left| \mathcal{N}_i \right|}$ if $(i.j) \in  \mathcal{E}$, else $a_{ij}=0$. The normalized laplacian matrix $\mathcal{L}$ of the graph $\mathcal{G}$ is defined as $\mathcal{L}=I-\mathcal{A}$.
%
%In the following, $\left\| \cdot \right\|$ denotes the Euclidian norm for vectors and the induced 2-norm for matrices. The minimum and maximum eigenvalues of a symmetric matrix $A$ are denoted by $\lambda_{min}(A)$ and $\lambda_{max}(A)$ respectively, and $A \succeq 0$ denotes a positive semidefinite matrix. 
%
% %The superscript $T$ means the transpose for real matrices, $\mathbb{R}^p$ denotes the $p$-dimensional Euclidian case. %, $\mathbb{R}^p$ denotes the $p$-dimensional Euclidian case, 
%%For any symmetric square matrix $A  \in \mathbb{R}^{m \times m}$ is the set of $m \times m$ real matrices. 
%%The minimum and maximum eigenvalues of a symmetric matrix $A$ are denoted by $\lambda_{min}(A)$ and $\lambda_{max}(A)$ respectively. %$A  \succ  0$ denotes a positive definite matrix. % and $A \succeq 0$ denotes a positive semidefinite matrix. The matrix $A^{-1}$ denotes the inverse of a square matrix $A$.
%In order to simplify the notation, we will only consider the case $p = 1$ in the following sections. The analysis can be extended for any dimension $p$ rewriting the equations in terms of Kronecker product.
%%The minimum and maximum eigenvalues of a symmetric matrix $A$ are denoted by $\lambda_{min}(A)$ and $\lambda_{max}(A)$ respectively. $A  \succ  0$ denotes a positive definite matrix. %and $A \succeq 0$ denotes a positive semidefinite matrix. 
%%the matrix $A^{-1}$ denotes the inverse of a square matrix $A$.


\subsection{Closed Level Curves}


Consider a scenario which can be described by a scalar field such as a space with varying concentration levels of radiation,  temperature, pressure or a toxic substance. 
%Consider a twice continuously differentiable source scalar field $\psi= \psi(r)$,  described by a mapping  $\psi: \mathbb{R}^p \rightarrow  \mathbb{R}^+$, where $p=1, 2,$ or $3$  and $r \in \mathbb{R}^p$ define the agents' position  in the space, e.g., for $p=3$, $r=[r_x \; r_y \; r_z ]^T$. 
%The scalar field  $\psi(r) \leq \psi(r^*)$ is bounded and it has a maximum value at position $r^*$, i.e., $\psi(r^*)= L_{\psi}$. 
%Let $p$ be the dimension of the space in which the agents move ($p=1$, $p=2$ or $p=3$); let $\psi:\mathbb{R}^p \rightarrow \mathbb{R}^+$ be a function that is twice continuously differentiable and which has an isolated local maximum $\psi(r_s)=L_{\psi}$ at $r_s$, with $\nabla \psi(r_s)=0$ and $\nabla^2 \psi(r_s) \prec 0$. We assume that there exist constants $L_{\nabla}$ and $L_H$ such that %$ \|  \psi(r) - \psi(\bar{r}) \| \leq L_{\nabla} \| r- \bar{r} \| $ and  $ \| \nabla \psi(r) - \nabla \psi(\bar{r}) \| \leq L_{H} \| r- \bar{r} \| $, $\forall r,\ \bar{r} \in \mathbb{R}^p$.
%$ \|  \psi(y) - \psi(x) \| \leq L_{\nabla} \| y- x \| $ and  $ \| \nabla \psi(y) - \nabla \psi(x) \| \leq L_{H} \| y - x \| $, $\forall \ y,\ x \in \mathbb{R}^p$.
%
%
%Let $\psi(r)$ be an unknown smooth scalar function in the plane. 
%
%---
%Let $p$ be the dimension of the space in which the agents move ($p=1,2,3$); let $r \in \mathbb{R}^p$ be the position of a single agent in the dimension $p$; let $\psi(r):\mathbb{R}^p \rightarrow \mathbb{R}$ be an unknown smooth scalar function that is twice continuously differentiable. Suppose we are given a family of closed regular curves $\mathcal{C}(\psi)$ with $\psi$ function in $\mathbb{R}^p$.
In order to solve the \textbf{level curve tracking problem} which has been defined in Chapter \ref{ch:introduction}, the following assumption has been made, and a family of closed regular curves $\mathcal{C}(\psi)$ with $\psi$ function in $\mathbb{R}^p$ is given.
 
\begin{assumption}\label{as:clc}
The function $\psi(r)$ is a $C^p$ smooth function on some bound open set $B \subset \mathbb{R}^p$. There exists a set of closed curves $\mathcal{C}(\psi)$; the level curves of $\psi$. On the set $B$ it is true that $\|\nabla \psi(r)\|\neq 0$ \citep{ZhangLeonard06, WilliamsSukhatme12}.
\end{assumption}

%\textcolor{red}{what about the maximum?.  ... Problem statement? revisar el capitulo 1!!}.

%\subsection{Problem statement}
%
%In this chapter, we consider multi-agent systems consisting of $N$ agents moving in the space $\mathbb{R}^p$ with a communication graph $\mathcal{G}$.  
%The problem  is to  find a distributed control law $u_i(t)$ that allows $N$ agents to track a level curve of an unknown scalar field $\psi(r)$ based on local information and  simultaneously, keeps agents moving at a common speed while maintaining a desired formation.

%-------------------------
%
%\section{Gradient estimation} \label{sec:lc_gradient}
%
%As proposed in \citep{RoseroWerner14b} and \citep{RoseroWerner14c}, the estimated gradient $\tilde{g}_i$ in each agent $i$ is computed in a cooperative way using only the neighboring information. Here the unweighted gradient approach with or without noise is used. 
%Consider mobile agents of size $N$, spatially distributed with communication graph $\mathcal{G}$. 
%Each agent $i$ measures the concentration $\psi(r_i)$ at its position $r_i$ where $i=1,2,...,N$. For each agent $i$, the concentration signal $\psi(r_i)$ at its position $r_i$ can be approximated by Taylor series. Neglecting high order terms and using the neighboring information $\mathcal{N}_i$, the concentration $\psi(r_i)$ can be written as
%%
%%
%%Since $\psi(r_i)$ is twice continuously differentiable, it can be  approximated by a Taylor series. 
%%%If the neighbor's position $r_j$ is close to the position of agent $r_i$, 
%%The value at $r_i$ is given by
%%\begin{align}\label{eq:gradient}
%%\psi(r_j) = & \psi(r_i) + (r_j-r_i)^T \nabla \psi(r_i) \nonumber \\
%%                & + \frac{1}{2} (r_j-r_i)^T \nabla^2 \psi(r_i) (r_j-r_i)  + H.O.T.,
%%\end{align}
%%where $i$ denotes the reference agent's index, $j$ denotes the neighbor agent that is sending information to agent $i$,  $\nabla \psi(r_i) \in \mathbb{R}^{p \times 1}$ is the true gradient,  $\nabla^2 \psi(r_i) \in \mathbb{R}^{p \times p}$ defines the Hessian, and $H.O.T.$ the higher order terms.
%%
%%Consider $N$ agents in a defined formation. 
%%Each agent $i$ measures the position $r_i \in \mathbb{R}^p$ and the signal strength $\psi(r_i)$ at its position $r_i$, where $i=1,2,\ldots,N$. 
%%Since $\psi(r_i)$ is $C^2$, it can be locally approximated by Taylor series as
%%\begin{align}\label{eq:gradient}
%%\psi(r_j) = & \psi(r_i) + (r_j-r_i)^T \nabla \psi(r_i) \nonumber \\
%%                & + \frac{1}{2} (r_j-r_i)^T \nabla^2 \psi(r_i) (r_j-r_i)  + H.O.T.,
%%\end{align}
%%where $i$ denotes the index of a reference agent, $j$ denoting the neighbor agents that are sending information to agent $i$,  $\nabla \psi(r_i)\in \mathbb{R}^p$ is the true gradient for the agent $i$, $\nabla^2 \psi(r_i) \in \mathbb{R}^{p \times p}$ define the Hessian matrix, and $H.O.T.$ represent higher order terms.
%%
%%\subsection{Unweighted gradient}
%%
%%For each agent $i$, it is possible to compute the slope between agent $i$ and their neighbors $\mathcal{N}_i$. Neglecting high order terms, equation (\ref{eq:gradient}) can be combined into
%\begin{align}\label{eq:unweightedgrad}
%\underbrace{
%\begin{bmatrix} 
%\psi(r_1) - \psi(r_i) \\
%%\psi(r_2) - \psi(r_i) \\
%\vdots	\\
%\psi(r_{\left| \mathcal{N}_i \right|}) - \psi(r_i)
%\end{bmatrix}
%}_{\mathsf{b}_i}
%=  &
%\underbrace{
%\begin{bmatrix}  
%(r_1-r_i)^T  \\
%%(r_2-r_i)^T  \\
%\vdots	 \\
%(r_{\left| \mathcal{N}_i \right|}-r_i)^T  
%\end{bmatrix}
%}_{\mathsf{R}_i}
%\nabla \psi(r_i),
%\end{align}
%where 
%$\nabla \psi(r_i) \in \mathbb{R}^{p \times 1}$ is the true gradient, 
%$\mathsf{b}_i   \in \mathbb{R}^{\left| \mathcal{N}_i \right| \times 1} $ is a column vector of the relative concentration value of the scalar field $\psi(r)$, 
%$\mathsf{R}_i   \in \mathbb{R}^{\left| \mathcal{N}_i \right| \times p} $ is a matrix whose coefficients depend on the relative position of the formation's geometric shape in the space $\mathbb{R}^p$. 
%%where $\mathsf{b}_i \in \mathbb{R}^{\left| \mathcal{N}_i \right| \times 1} $, 
%%$\mathsf{R}_i \in \mathbb{R}^{\left| \mathcal{N}_i \right| \times p} $.
%The problem can be solved  minimizing  $\left\| \mathsf{R}_i \tilde{g}_i - \mathsf{b}_i \right\|^2$, then 
%\begin{align}\label{eq:estgradient}
%\tilde{g}_i = & \left( \sum_{j \in \mathcal{N}_i} a_{ij} (r_j - r_i)(r_j - r_i)^T  \right)^{-1}  \nonumber\\
%          & \times \left( \sum_{j \in \mathcal{N}_i} a_{ij} (r_j - r_i) ( \psi(r_j) - \psi(r_i) ) \right), \nonumber \\
%          =& \left ( \mathsf{R}_i^T \mathsf{R}_i\right )^{-1} \mathsf{R}_i^T \mathsf{b}_i,
%\end{align}
%where 
%$\tilde{g}_i \in \mathbb{R}^{p \times 1}$ is the distributed estimated gradient provided that the inverse of $\mathsf{R}_i^T \mathsf{R}_i$ exists. This means the matrix $\mathsf{R}_i$ must be full column rank (rank($\mathsf{R}_i$)$=p$). 
%%The rank ($\mathsf{R}_i$) $=p$ if and only if the agent $i$ and their neighbors $\mathcal{N}_i$ are non collinear, i.e., they do not collapse in a plane ($p=3$), in a line ($p=2$) or in a point ($p=1$). 
%Since $\mathsf{R}_i \in \mathsf{R}^{\left| \mathcal{N}_i \right| \times p} $, this requires that $\left| \mathcal{N}_i \right| \geq p $.  In order to estimate the gradient, the minimum number of  agents is $N_{min} = p+1$.
%
%%Neglecting the higher order terms, the estimation error for each agent $i$ can be computed as
%%\begin{align}
%%e_{\nabla i} =& (\mathsf{R}_i^T \mathsf{R}_i )^{-1} \mathsf{R}_i^T e_{di},
%%%e_{\nabla i} =& \hat{g}_{i} - \nabla \psi(r_i) = (\mathsf{R}_i^T \mathsf{R}_i )^{-1} \mathsf{R}_i^T e_{di},
%%\end{align}
%%where
%%\begin{align*}
%%e_{di} &= 
%%\begin{bmatrix}  
%%\frac{1}{2} (r_1-r_i)^T \nabla^2 \psi(r_i) (r_1-r_i) \\
%%%\frac{1}{2} (r_2-r_i)^T \nabla^2 \psi(r_i) (r_2-r_i)  \\
%%\vdots	 \\
%%\frac{1}{2} (r_{\left| \mathcal{N}_i \right|}-r_i)^T \nabla^2 \psi(r_i) (r_{\left| \mathcal{N}_i \right|}-r_i)
%%\end{bmatrix},
%%%+
%%%\begin{bmatrix}
%%%e_{M1}\\
%%%e_{M2}\\
%%%\vdots  \\
%%%e_{M \left| \mathcal{N}_i \right|}
%%%\end{bmatrix},
%%\end{align*}
%%with $e_{di} \in \mathbb{R}^{\left| \mathcal{N}_i \right|\times 1}$. % and the vector $[e_{M1},\ldots, e_{M \left| \mathcal{N}_i \right|}]^T$ is the relative measurement noise. 
%%Defining the maximum distance between agents in the formation as $h = \text{max}_{j \in \mathcal{N}_i } \| r_j - r_i \|$ and considering the Hessian $\nabla^2 \psi(r_i)$ is bounded by $\|\nabla^2 \psi(r_i)\| \leq L_H$, then
%%$
%%\frac{1}{2} (r_j-r_i)^T \nabla^2 \psi(r_i) (r_j-r_i) \leq \frac{1}{2} L_{Hi} \|r_j-r_i\|^2 \leq \frac{1}{2} L_{Hi} h^2.
%%$
%%Thus, for each agent $i$, the error $e_{di}$ is bounded by 
%%$
%%\|e_{di}\| \leq \frac{1}{2} L_{Hi} h^2 \sqrt{\left| \mathcal{N}_i \right|}
%%$, 
%%and the estimation error by
%%\begin{align}
%%\|e_{\nabla i}\| \leq & \|(\mathsf{R}_i^T \mathsf{R}_i)^{-1}\mathsf{R}_i^T\|\|e_{di} \|  \nonumber \\
%% \leq & \frac{1}{2} L_{Hi} h^2 \sqrt{\left| \mathcal{N}_i \right|} \|(\mathsf{R}_i^T \mathsf{R}_i)^{-1}\mathsf{R}_i^T\| = e_{0i}.
%%\end{align}
%%Note that the estimated error depends on the square of the distance between  agents. To make the lower order terms in the Taylor expansion dominate, the distances between agents $\left\| r_j - r_i \right\|$ must be sufficiently small.
%
%
%To estimate the gradient when the concentration and position signals of each agent $i$ are corrupted by noise, a distributed consensus algorithm developed in \citep{RoseroWerner14a} is used. This consensus filter attenuates high frequency noise and improve the accuracy of the estimated signals. Assume that each agent measures either concentration signal $\psi_i(r_i(t)) \in \mathbb{R}$ or position signal $r_i(t) \in \mathbb{R}^p$ corrupted by noise $n_{\psi i}(t) \in \mathbb{R}$ or $n_{r i}(t) \in \mathbb{R}^p$ respectively.
%The sensing models are given by
%\begin{align}
%u_{\psi i}(t) & = \psi_i(t) + n_{\psi i}(t),\nonumber \\
%u_{r i}(t)     & = r_i(t) +  n_{r i}(t),
%\end{align}
%where $n_{\psi i} \in \mathbb{R}$ and $n_{r i} \in \mathbb{R}^p$ are zero-mean Gaussian noise, and the pairs $n_{r i}(t)$ and $n_{r j}(t)$, and $n_{\psi i}(t)$ and $n_{\psi j}(t)$ are uncorrelated. Based on \citep{RoseroWerner14a} the estimated concentration signal of each agent $i$ can be written as
%% In order to estimate the signals, a distributed consensus algorithm based on \citep{RoseroWerner14a} is used. To estimate the concentration signal, we propose
%\begin{align} \label{eq:concentrationest}
%\dot{\xi}_i(t) = & \beta_{\xi} \sum_{j \in \mathcal{N}_i} a_{ij}\left( u_{\psi i}(t)-u_{\psi j}(t)\right) \nonumber \\
% &  - \beta_{\xi} \sum_{j \in \mathcal{N}_i} a_{ij} (\xi_i(t)-\xi_j(t)) \nonumber \\
% & +\beta_{\xi}(1+d_i)\left(u_{\psi i}(t)-\xi_i(t) \right),
%\end{align}
%where $\xi_i \in \mathbb{R}$ is the target's estimation $\psi_i$ and $\beta_{\xi}$ is a tuning parameter. The estimated position signal of each agent $i$ can be written as
%\begin{align} \label{eq:positionest}
% \dot{\nu}_i(t) = & \beta_{\nu} \sum_{j \in \mathcal{N}_i} a_{ij}\left( u_{r i}(t)-u_{r j}(t)\right) \nonumber \\
% & - \beta_{\nu} \sum_{j \in \mathcal{N}_i} a_{ij} (\nu_i(t)-\nu_j(t)) \nonumber \\
% & +\beta_{\nu} (1+d_i)\left(u_{r i}(t)-\nu_i(t) \right),
%\end{align}
%where $\nu_i \in \mathbb{R}^p$ is the estimation of the target $r_i$ and $\beta_{\nu}$ is a tuning parameter.   
%%Define the vectors $u_{\psi}=[u_{\psi 1},\ldots,u_{\psi N}]^T$, $n_{\psi}=[n_{\psi 1},\ldots,n_{\psi N}]^T$, and $u_{r}=[u_{r 1}^T,\ldots,u_{r N}^T]^T$,  $n_{r}=[n_{r 1}^T,\ldots,n_{r N}^T]^T$. 
%%In vector notation, and using the definition of Laplacian graph, equation (\ref{eq:concentrationest}) can be written as
%%\begin{align}
%%\dot{\xi}(t)=-\beta_{\xi}A(\xi-u_{\psi})
%%\end{align}
%%and equation (\ref{eq:positionest}) can be written as
%%\begin{align}
%%\dot{\nu}(t)=-\beta_{\nu}A(\nu-u_{r})
%%\end{align}
%%where $A=I_N +\Delta +\mathcal{L}$ is a symmetric positive definite matrix. 
%Each state of sensor $i$ is initialized with $\xi_i(0)=\xi_{i0}$ and $\nu_i(0)=\nu_{i0}$.   
%
%Combining the  estimated concentration and position signals (outputs of consensus filters) and using only the neighboring information $\mathcal{N}_i$ of agent $i$ is possible to write
%%Using the outputs of consensus filter, it is possible to compute the slope between agent $i$ and their neighbors $\mathcal{N}_i$ as
%%\begin{align}
%%\hat{g}_{e i}(t) = \left(\sum_{j \in \mathcal{N}_i}a_{ij}(z_i-z_j)(z_i-z_j)^T\right)^{-1} \nonumber \\
%%\times \left( \sum_{j \in \mathcal{N}_i}a_{ij}(z_i - z_j)(\xi_i-\xi_j) \right)
%%\end{align}
%\begin{align}\label{eq:weightednoisegrad}
%%\mathsf{W}_i
%\underbrace{
%\begin{bmatrix} 
%\xi(\nu_1) - \xi(\nu_i) \\
%%\xi(\nu_2) - \xi(\nu_i) \\
%\vdots	\\
%\xi(\nu_{\left| \mathcal{N}_i \right|}) - \xi(\nu_i)
%\end{bmatrix}
%}_{\mathsf{b}_i}
%=  &
%%\mathsf{W}_i
%\underbrace{
%\begin{bmatrix}  
%(\nu_1-\nu_i)^T  \\
%%(\nu_2-\nu_i)^T  \\
%\vdots	 \\
%(\nu_{\left| \mathcal{N}_i \right|}-\nu_i)^T  
%\end{bmatrix}
%}_{\mathsf{R}_i}
%\nabla \xi(\nu_i),
%%    \nonumber\\
%%&+
%%\frac{1}{2} M
%%\underbrace{
%%\begin{bmatrix}  
%%\| r_1 -r_i \|^2  \\
%%\| r_2 -r_i \|^2 \\
%%\vdots	 \\
%%\| r_{\left| \mathcal{N}_i \right|}  - r_i \|^2 
%%\end{bmatrix}
%%}_{e_{di}}
%\end{align}
%where 
%$\mathsf{b}_i   \in \mathbb{R}^{\left| \mathcal{N}_i \right| \times 1} $, 
%$\mathsf{R}_i \in \mathbb{R}^{\left| \mathcal{N}_i \right| \times p} $ and 
%$\nabla \xi(\nu_i)   \in \mathbb{R}^{p \times 1} $. 
%By least-squares estimator the estimated gradient can be computed as
%\begin{equation}\label{eq:noisegradest}
%\tilde{g}_i=\tilde{g}_{ei}
%= \left( \mathsf{R}_i ^T  \mathsf{R}_i \right )^{-1}
% \mathsf{R}_i^T  \mathsf{b}_i.
%\end{equation}
%% The stability analysis of these consensus filters can be separated because  filters' dynamics are faster than  agents' dynamics,  and  filters' dynamics have no interaction with  agents' states. 
%% A consensus filter's stability analysis has been presented in \citep{RoseroWerner14a}. 
%
%
%\begin{assumption} 
%Graph $\mathcal{G}$ is undirected and connected. In order to compute the estimated gradient  $\tilde{g}_i$ by least-squares, we assume the formation satisfies $ \left| \mathcal{N}_i \right| \geq p $, $N \geq p+1$ and the agent $i$ and their neighbors are not collinear, i.e., each agent has at least $p$ neighbors and the minimum number of the agents in the space is $p+1$ \citep{RoseroWerner14b, RoseroWerner14c}.
%\end{assumption}
%
%\begin{assumption}
%%The scalar field has an isolated local maximum. 
%The estimated gradient is bounded by $\| \tilde{g}_i(r_i) \| \leq \mu$, i.e., the estimated gradient will not be infinite for any time $t$. The average of the estimated gradient's error is bounded by $\|\bar{e}_{\nabla}\| \leq e_0$ \citep{RoseroWerner14b, RoseroWerner14c}. 
%\end{assumption}
%
%---------------------------------

\subsection{Problem Statement}

The \textbf{ level curve tracking problem} is defined as follows: Find a distributed control law $u_i(t)$ that allows $N$ agents to track a level curve into an unknown scalar field $\psi(r)$. % based on local information and  simultaneously, keeps agents moving at a common speed while maintaining a desired formation.

%%%%%%%%%%%%%%%%%%%%%%%%%%%%%%%%%%%%%%%%%%%%%%%%%%%%%%%%%

Next, the control laws for double integrator agents and quadcopter agents are presented.


\section{Cooperative Level Curve Tracking for Double Integrator Agents}  \label{sec:lc_doubleint}

The dynamics of each agent $i$ in a network modelled as double integrator are described by
\begin{align} \label{eq:lct_di}
\dot{r}_i(t)& = v_i(t), \nonumber \\
\dot{v}_i(t) & = u_i(t),
\end{align}
where %$i \in \mathcal{V}$ or 
$i=1,...,N$, $r_i(t) \in \mathbb{R}^{p}$ is the position vector, $v_i \in \mathbb{R}^{p}$ is the velocity vector, and $u_i(t) \in \mathbb{R}^{p}$ is the acceleration's  control input. 
To solve the level curve tracking problem,  a distributed control law consisting of two parts is presented: the two parts are, first, a formation control law $u_{Fi}$ and second, a trajectory control law $u_{Ti}$ as 
%\begin{equation} \label{eq:lct_di_dcl}
%\mbox{$\small
%\begin{split}
%u_i(t) = & u_{Fi}(t) +u_{Ti}(t), \\
%u_{Fi}(t) = & k_F \sum_{j \in \mathcal{N}_i} a_{ij} [ (r_{Fi}(t)-r_{Fj}(t)) -(r_i(t)-r_j(t)) - \theta (v_i(t)-v_j(t)) ], \\
%u_{Ti}(t)= & k_{\psi}(\psi_{ref}-\psi_i)\hat{g}_i(t)+k_T\left(T\hat{g}_i(t) -\frac{\gamma}{k_T} v_i(t) \right),
%\end{split}
%$}
%\end{equation}
\begin{align} \label{eq:lct_di_dcl}
u_i(t) = & u_{Fi}(t) +u_{Ti}(t), \nonumber \\
u_{Fi}(t) = & k_F \sum_{j \in \mathcal{N}_i} a_{ij} [ (r_{Fi}(t)-r_{Fj}(t)) -(r_i(t)-r_j(t)) - \theta (v_i(t)-v_j(t)) ], \nonumber \\
u_{Ti}(t)= & k_{\psi}(\psi_{ref}-\psi_i)\frac{\hat{g}_i}{\|\hat{g}_i\|}+k_T\left(T\frac{\hat{g}_i}{\|\hat{g}_i\|}-\frac{\gamma}{k_T} v_i(t) \right),
\end{align}
where
%$\tilde{g}_i=\frac{\hat{g}_i}{\|\hat{g}_i\|}$, 
$r_{Fi}=Q(\alpha _i)\tilde{r}_{Fi}$, 
$
Q(\alpha_i)=
\begin{bmatrix}
\cos(\alpha_i) & \sin(\alpha_i)\\
-\sin(\alpha_i) &  \cos(\alpha_i)
\end{bmatrix}
$
 rotates the agent $i$ accordingly to the gradient's direction as 
$\alpha_i= \tan^{-1}\left(\frac{\hat{g}_{yi}}{\hat{g}_{xi}}\right)$, $\tilde{r}_{Fi}$ is the desired position. 
$u_{Fi}(t)$ keeps agents in their desired relative positions, %denotes the formation control input for agent $i$, 
$r_{Fi}(t)$ denotes a formation's desired reference signal for agent $i$, 
$u_{Ti}(t)$ drives the whole group of agents to a desired level curve of the scalar field, %denotes the control input tracking the gradient for agent $i$,
$k_{\psi}>0$, $k_F>0$, $k_T>0$, $\gamma >0$, $\theta > 0$ are scalar control parameters in the formation  
and $\tilde{g}_i(t)$ is the estimated gradient for agent $i$ computed by equations (\ref{eq:estgradient}) or (\ref{eq:noisegradest}). Since level curve tracking implies movement towards the orthogonal  direction of the gradient, a rotation matrix $T$ is chosen for counterclockwise rotation as 
$
T=
\begin{bmatrix}
0 & -1 \\
1  & 0
\end{bmatrix},
$
or clockwise rotation $-T$ in the case of $\mathbb{R}^2$.  The term $k_{\psi}(\psi_{ref}-\psi_i(t))\frac{\hat{g}_i}{\|\hat{g}_i\|}$ depends on the difference between current and desired level curve and moves agents in the gradient direction. For this reason, when agents are far from a desired level curve the gain of this term is large and cause fast movements;  while if agents are close to a desired level curve the gain will be smaller.

%
%As presented in \citep{RoseroWerner14b}, the source seeking problem is solved based on the distributed control law
%%\small
%\begin{align}\label{eq:lct_di_ssp}
%%u_i(t) = & u_{Fi}(t) +u_{Ti}(t), \\
%u_{i}(t) = & k_F \sum_{j \in \mathcal{N}_i} a_{ij} [ (r_{Fi}(t)-r_{Fj}(t)) -(r_i(t)-r_j(t)) \nonumber \\
% & - \theta (v_i(t)-v_j(t)) ]+k_T\left(\hat{g}_i(t) -\frac{\gamma}{k_T} v_i(t) \right), 
%\end{align}
%%\normalsize
%and  agents locate the scalar field's source moving to the maximum while the geometric formation is maintained. \textcolor{red}{The difference is ....} 

Let  $r(t) = [ r_1^T(t) , \ldots, r_N^T(t) ]^T$,  
$r_F(t) = [ r_{F1}^T(t) ,\ldots, r_{FN}^T(t)]^T$, \\ 
$v(t) = [ v_1^T(t) ,\ldots, v_N^T(t) ]^T$, and
$\hat{g}(t) = \left[ \hat{g}^T_1(t),\ldots, \hat{g}^T_N(t) \right]^T$, 
the closed-loop dynamics can be written as
\begin{align}\label{eq:lct_secondorder}
\begin{bmatrix}
\dot{r}\\
\dot{v}
\end{bmatrix}
= 
\Sigma
\begin{bmatrix}
r\\
v
\end{bmatrix}
+
\begin{bmatrix}
0 & 0 \\
k_F\mathcal{L} & I_N  
\end{bmatrix}
\begin{bmatrix}
r_F \\
u_T
\end{bmatrix},
\end{align}
where
\begin{align*}
\Sigma=\begin{bmatrix}
0 & I_N \\
-k_F \mathcal{L}  &-k_F \theta \mathcal{L} - \gamma I_N
\end{bmatrix},
\end{align*}
\begin{align*}
u_{T}= & k_{\psi}f(t)+k_Th(t),
\end{align*}
and 
$f(t)=P(\psi)\frac{\hat{g}_i}{\|\hat{g}_i\|}$, 
$h(t)=\bar{T}\frac{\hat{g}_i}{\|\hat{g}_i\|}$, 

$P(\psi)= diag\left[\psi_{ref}-\psi_1, \ldots, \psi_{ref}-\psi_N \right]= P(\psi)\otimes I_p$, 

$\bar{T}= diag\left[T, \ldots, T\right]=I_N \otimes T$. 

Note that if $\hat{g}(r)=0$, $k_T =1$ and $k_F=1$, equation (\ref{eq:lct_secondorder}) is reduced to the well known results on agents' formation  presented by  \citep{RenAtkins07}.  
%where
%\begin{align*}
%\Gamma=\begin{bmatrix}
%0 & I_N \\
%-\mathcal{L}  &-\theta \mathcal{L} 
%\end{bmatrix}.
%\end{align*} 
%In  \citep{RenAtkins07}, it is shown that consensus is reached asymptotically  if and only if $\Gamma$ has exactly two zero eigenvalues and all the other eigenvalues have negative real parts. In case of undirected connected graph $\mathcal{G}$, all eigenvalues are real, and thus, by Lemma 4.2 of  \citep{RenAtkins07}, it follows that consensus is achieved for all $\theta >0$.


Let  $\bar{r}(t)=\frac{1}{N} \sum_{i \in \mathcal{V}} r_i(t)= \frac{1}{N} \boldsymbol{1}^T r(t)$ be the average of the position states, $
\bar{v}(t)=\frac{1}{N} \sum_{i \in \mathcal{V}} v_i(t)= \frac{1}{N} \boldsymbol{1}^T v(t)$ be the average of the velocity states, and $
\bar{\hat{g}}(r(t))=\frac{1}{N} \sum_{i \in \mathcal{V}} \hat{g}_i(r(t))= \frac{1}{N} \boldsymbol{1}^T \hat{g}(r(t))$ be the average of the estimated gradient.  Since $\boldsymbol{1}^T \mathcal{L}=\boldsymbol{0}^T$, the time derivative of  both the position states  average $\bar{r}$  and the velocity states  average $\bar{v}$ is given by
\begin{align}
\dot{\bar{r}}(t) & = \bar{v}(t), \nonumber \\
\dot{\bar{v}}(t) & = -\gamma\bar{v}(t) + \bar{u}_T,
\end{align} 
with initial average $\bar{r}(0)=\frac{1}{N}\boldsymbol{1}^Tr(0)=\bar{r}_{0} $, $\bar{v}(0)=\frac{1}{N}\boldsymbol{1}^Tv(0)=\bar{v}_{0} =0 $ and $\bar{\hat{g}}(r(0))= \frac{1}{N}\boldsymbol{1}^T\hat{g}(r(0))=\bar{\hat{g}}_{0}$. 
 Note that agents move inside a scalar field with varying velocity  in the gradient direction. %If the estimated gradient $\tilde{g}_i$ for each agent $i$ is normalized as $\hat{g}_i=\frac{\tilde{g}_i}{\|\tilde{g}_i\|}$, agents' velocity  will be  constant. 
%Similar analysis for cooperative source seeking

%With the purpose of analysing the consensus of equation (\ref{eq:lct_secondorder}), first the equation $\textsl{det} \left(\sigma I_{2N}-\Sigma \right)=0$ is solved to find the eigenvalues of $\Sigma$, where $\textsl{det} \left(\sigma I_{2N}-\Sigma \right)$ is the characteristic polynomial of matrix $\Sigma$. 
%Based on \citep{RenAtkins07}, it is true that
%\begin{align*}
%\textsl{det}\left( \sigma I_{2N}-\Sigma\right)& = \textsl{det} \left( \sigma^2I_N + (\gamma I_N + k_F \theta \mathcal{L})\sigma+k_F\mathcal{L} \right), %\\
%%& = \prod_{i=1}^N \left( \sigma^2 + (k_T\gamma  - k_F\theta \lambda_i)\sigma - k_F\lambda_i \right).
%\end{align*}
%and the eigenvalues of $\Sigma$  are given by
%\begin{align*}
%\sigma_{i1,\ i2}=\frac{ -(\gamma - k_F \theta \lambda_i) \pm \sqrt{(\gamma - k_F \theta \lambda_i)^2 + 4k_F\lambda_i }}{2}, 
%\end{align*}
%where $\sigma_{i1}$ and $\sigma_{i2}$ are called eigenvalues of  $\Sigma$ that are associated with $\lambda_i$ of Laplacian matrix $\mathcal{L}$. 
%%It is known by \citep{OlfatiMurray04} and \citep{RenAtkins07} that $\mathcal{L}$ has a simple zero eigenvalue and all the other eigenvalues have positive real parts if and only if the digraph has a spanning tree. If the graph is undirected, then the eigenvalues of $-\mathcal{L}$ are all real numbers and $\lambda_i<0$ ($i=2,3,\ldots,N$) if and only if the graph is connected. The algebraic multiplicity of the eigenvalue zero is one. 
%%Lemma 4.6 of \citep{RenAtkins07} and lemma 1 of \citep{ZhuTianKuang09} revealed that consensus is achieved if and only if all roots of $\sigma^2 + (k_T\gamma  - k_F \theta \lambda_i)\sigma - k_F\lambda_i$ with $i=2,3,\ldots,N$, have negative real parts.
%%Thus for each $\lambda_i$, there exists two eigenvalues of $\Sigma$, denoted by $\sigma_{i1}$ and $\sigma_{i2}$ respectively.  
%Since $\lambda_1=0$,  two eigenvalues $\sigma_{11} =0$ and $\sigma_{12}=-\gamma$ are obtained. Then, $\Sigma$ has one zero  eigenvalue because $\mathcal{L}$ has one zero eigenvalue, and all the other eigenvalues have negative real parts. 
%%Let $q=[q_a^T \ q_b^T]^T$ be an eigenvector o $\Sigma$ associated with eigenvalue zero, where $q_a, \ q_b \ \in \mathbb{R}^N$. Then $\Sigma q =\boldsymbol{0}$, which implies that $q_b=\boldsymbol{0}$ and $-\mathcal{L}q_a = \boldsymbol{0}$. That is, $q_a$ is an eigenvector of $-\mathcal{L}$ associated with eigenvalue zero, and $q=[q_a^T \ \boldsymbol{0}^T]^T$. 
%
%Lemma 4.6 of \citep{RenAtkins07} reveals that consensus is achieved if and only if matrix $\Sigma$ has a simple zero eigenvalue and all the other eigenvalues have negative real parts. Let $\mathsf{p}$ be a non-negative left eigenvector of $\mathcal{L}$ associated with eigenvalue $0$, $\mathsf{p}^T\mathcal{L}=0$ and $\mathsf{p}^T\boldsymbol{1}=1$. Without loss of generality,  $\mathsf{w}_1=[\boldsymbol{1}^T \ \boldsymbol{0}^T]^T$  is chosen as the right eigenvector and  $\mathsf{v}_1=[\mathsf{p}^T \ \frac{1}{\gamma }\mathsf{p}^T]^T$ is chosen as the left eigenvector corresponding to the eigenvalue zero, where $\mathsf{v}_1^T \mathsf{w}_1=1$ and $\mathsf{p}= \frac{1}{N}\boldsymbol{1}$. Then, given any initial positions and velocities $r(0)=r_0$ and $v(0)=v_0$, respectively, it follows that 
%\begin{align*}
%\begin{bmatrix}
%r(t)\\
%v(t)
%\end{bmatrix}
%\rightarrow
%\frac{1}{N}
%\begin{bmatrix}
%\boldsymbol{1} \boldsymbol{1}^T & \frac{1}{\gamma} \boldsymbol{1} \boldsymbol{1}^T \\
%0 & 0
%\end{bmatrix}
%\begin{bmatrix}
%r(0)\\
%v(0)
%\end{bmatrix},
%\end{align*}
%then $\lim_{t\rightarrow \infty} r(t) = \frac{1}{N}\boldsymbol{1}\boldsymbol{1}^Tr(0)+\frac{1}{\gamma N}\boldsymbol{1}\boldsymbol{1}^Tv(0)$ and $\lim_{t\rightarrow \infty} v(t) = 0$.  It is straightforward to see  that consensus is achieved if and only if $k_F>0$, $k_T>0$, $\gamma >0$, $\theta > \textsf{max}_{2\leq i \leq N}\frac{\gamma}{k_F\lambda_i}$,  and the graph is connected. 
%%Similar results are obtained applying  a procedure  based on \citep{Zhu11}.
%%
%%Theorem 4.3  of \citep{RenAtkins07} revealed that the consensus protocol of equation (\ref{eq:secondorder}) with inputs equal to zero is given by $r(t) \rightarrow \boldsymbol{1}p^Tr(0)+\frac{k_F}{k_T\gamma}\boldsymbol{1}p^Tv(0)$, and $v(t) \rightarrow 0$ asymptotically  as $t \rightarrow0$ if and only if matrix $\Sigma$ has a simple zero eigenvalue and all the other eigenvalues have negative real parts???.
%
%Thus, matrix $\Sigma$ has exactly one zero eigenvalue and one only independent eigenvector corresponding to the zero eigenvalue. The eigenvalues of $\Sigma$ are denoted by $-\kappa_{2N}(\Sigma) \leq \dots \leq -\kappa_2(\Sigma) < \kappa_1(\Sigma)=0$. Then for all $t\geq 0$ and all vectors $\mathsf{v} \in \mathbb{R}^{2N}$ with  $\mathsf{w}_1^T\mathsf{v}=0$ and $\mathsf{v}_1^T\mathsf{v}=0$,  it holds that 
%\begin{align}\label{eq:lct_bounddi}
%\|e^{\Sigma t}\mathsf{v}\| \leq e^{-\kappa_2 t}c_d\|\mathsf{v}\|,
%\end{align}
%where $c_d=\|P^{-1}\| \|P\|$. 



The proposed distributed controller has two parts: formation controller and tracking controller. The formation controller maintains agents in a desired relative position and the tracking controller steers  agents in the perpendicular direction of the gradient in order to track a level curve.



%\subsection{Equilibrium point}




The equilibrium point of equation (\ref{eq:lct_secondorder}) is given by
\small
\begin{align}\label{eq:lct_doubleinteqpoint}
\dot{r}^*=&v^*, \nonumber\\
0 = &k_F \mathcal{L}(r_F-r^*)-(k_F\theta \mathcal{L}+\alpha I)v^*+k_{\psi}f^*+k_Th^*,
\end{align}
\normalsize
where $r^*=[r_1^{*T},\ldots,r_N^{*T}]^T$. Then the equilibrium points are given by
\begin{align}
r^*=&\int_0^tv^*d\tau, \nonumber\\
v^*=&\left(k_F\theta\mathcal{L}\right)^{-1}\left(kf\mathcal{L}(r_F-r^*)+k_{\psi}f^*+k_Th^*\right).
\end{align}

In order to solve the level curve tracking problem for double integrator agents, the following theorem is presented.


\begin{theorem}
Consider the multi-agent system (\ref{eq:lct_di}) with control law (\ref{eq:lct_di_dcl}). Suppose that the assumptions \ref{as:1}, \ref{as:2}, \ref{as:4} and \ref{as:clc} are fulfilled. Then, for all $r_i(0) \in \mathbb{R}^p $ and $t \geq 0$, agents track an unknown scalar field's level curve $\psi(r)$ and the disagreement vector  $\delta$ of the closed-loop system converges to a ball centred at the origin with radius
\begin{align}\label{eq:lct_radiusdi}
\epsilon=  \frac{2 c_d \sqrt{N}}{\kappa_2}(k_T+ak_{\psi}).
\end{align}
\end{theorem}
%\textcolor{red}{definir!!! $r*$}
%
%\begin{theorem}
%Consider a network of $N$ agents defined by (\ref{eq:di}), moving in a scalar field $\psi(r): \mathbb{R}^p \rightarrow \mathbb{R}^+$,  with an  undirected, connected communication graph $\mathcal{G}$. Let  the network and scalar field fulfill assumptions 1 to 2. If this kind of networks obey the  distributed control law (\ref{eq:di_dcl}), 
%then, for all $r_i(0) \in \mathbb{R}^p $ and $t \geq 0$,  the unknown static source's position $r_s$ of the scalar field $\psi(r)$ is an asymptotically stable equilibrium of the system  (\ref{eq:di}). It means the overall system will locate the unknown source $r_s$ with an error $\frac{2e_0 + L_H h}{2 \gamma}$ and maintain the desired formation with
%\small
%\begin{align*}
%\|\delta(t)\| \leq & \frac{c_d}{ \sigma_2} \left( 2k_T \sqrt{N}\mu \right) + \nonumber \\
%& \left( c_d\|\delta(0)\| - \frac{c_d}{\sigma_2} \left(2k_T \sqrt{N}\mu \right) \right) e^{-\sigma_2 t}.
%\end{align*}
%\normalsize
%%and the position's average of the agents $\bar{r}(\infty)$ has a bounded distance from the source $r^*$ less than $e_0$.
%Moreover if the agents travel a constant velocity, it holds that
%\small
%\begin{align*}
%\|\delta(t)\| \leq & \frac{c_d}{\sigma_2} \left( 2k_T \sqrt{N} \right) + \left( c_d\|\delta(0)\| - \frac{c_d}{\sigma_2} \left( 2k_T \sqrt{N} \right) \right) e^{-\sigma_2 t}.
%\end{align*}
%\normalsize
%\end{theorem}
%
%
%\
%\textbf{Remark:} Note that if  the estimated gradient is normalized, 
%$\hat{g}^N(t) = \left[ \frac{\hat{g}^T_1(t)}{\| \hat{g}_1(t) \|} \; ... \; \frac{\hat{g}^T_N(t)}{\| \hat{g}_N(t) \|} \right]^T$, and $\| \frac{\hat{g}_i}{\| \hat{g}_i \| }\| = 1$, then  $\| \hat{g}^N \| = \sqrt{N}$.
%Thus, the region of convergence is defined by
%\begin{align*}
%\|\delta(t)\| \leq & \frac{c_d}{\sigma_2} \left( 2 k_T \sqrt{N} \right) + \nonumber \\
%& \left( c_d\|\delta(0)\| - \frac{c_d}{\sigma_2} \left( 2 k_T \sqrt{N} \right) \right) e^{-\sigma_2 t}.
%\end{align*}


\begin{proof}
Consider the position error $e_p(t)=r(t)-r^*$  and velocity error $e_v(t)=v(t)-v^*$. The dynamic error can be computed as
\begin{align}\label{eq:lct_errordoubleint}
\begin{bmatrix}
\dot{e}_p\\
\dot{e}_v
\end{bmatrix}
= 
\Sigma
\begin{bmatrix}
e_p\\
e_v
\end{bmatrix}
+
\begin{bmatrix}
0 \\
k_{\psi}e_f+k_Te_h
\end{bmatrix},
\end{align}
where 
$e_f=f-f^*=P(\psi)\hat{g}(e_p+r^*)-P(\psi^*)\hat{g}(r^*)$,
$e_h=h-h^*=\bar{T}\left(\hat{g}(e_p+r^*)-\hat{g}(r^*)\right)$.  
Define average of position error states
$
\bar{e}_p(t)=\frac{1}{N}\sum_{i \in \mathcal{V}}e_{pi}(t)=\frac{1}{N} \boldsymbol{1}^Te_p(t) =\bar{r}-\bar{r}^*,
$
and the average of the velocity error states
$
\bar{e}_v(t)=\frac{1}{N}\sum_{i \in \mathcal{V}}e_{vi}(t)=\frac{1}{N} \boldsymbol{1}^Te_v(t)= \bar{v}-\dot{\bar{r}}^*.
$
%An important problem for multi-agent systems is the transit performance. The performance of the first-order consensus protocols was discussed by \citep{OlfatiMurray04} and it was revealed that if the graph is undirected, then the consensus speed is determined by the algebraic connectivity, i.e., the smallest nonzero eigenvalue of the Laplacian matrix of the undirected connected graph.  Results about the performance are given by \citep{Zhu11} for the second-order consensus protocol. The largest and the smallest nonzero eigenvalues of the connected graph's Laplacian matrix determine the maximum consensus speed. Theorem 1 of \citep{Zhu11} reveals that the maximum consensus speed is achieved with exponential decay $e^{-\upsilon t}$, where
%\begin{align}
%\upsilon = \sqrt{\frac{k_F\lambda_2\lambda_N}{\lambda_2-2\lambda_N}}.
%\end{align}
%
%By theorem 1 of \citep{Zhu11}, one can choose appropriate gains $k_F$ and $k_F\theta$ such that the error dynamics have the expected convergence speed as
%\begin{align}
%k_F > \frac{\kappa^2(\lambda_2-2\lambda_N)}{\lambda_2\lambda_N}
%\end{align}
%and 
%\begin{align}
% \theta = \theta^* = \frac{2\sqrt{-k_F\lambda_N}}{k_F \sqrt{-\lambda_2(\lambda_2-2\lambda_N)}}.
%\end{align}
Then, the state vector can be decomposed according to
\begin{align}\label{eq:lct_disagvector}
e_p(t) & = \boldsymbol{1}\bar{e}_p(t) + \delta_p(t),  \nonumber \\
e_v(t) & = \boldsymbol{1}\bar{e}_v(t) + \delta_v(t),
\end{align}
such that the disagreement vectors $\delta_p(t)$ and $\delta_v(t)$ have zero average, i.e., $\boldsymbol{1}^T \delta_p(t) \equiv \boldsymbol{1}^T \delta_v(t) \equiv 0$. Derivation of  (\ref{eq:lct_disagvector}) with respect to time $t$ and considering that $v(0)=0$, yields
\begin{align}
\begin{bmatrix}
\dot{\delta}_p\\
\dot{\delta}_v
\end{bmatrix}
= &
\Sigma 
\begin{bmatrix}
\delta_p\\
\delta_v
\end{bmatrix}
+
k_T
\begin{bmatrix}
0 & 0\\
k_{\psi} \mathcal{M} & k_T \mathcal{M}
\end{bmatrix}
\begin{bmatrix}
e_f\\
e_h
\end{bmatrix},
\end{align}
where $\mathcal{M} =  I_N -\frac{1}{N} \boldsymbol{1}\boldsymbol{1}^T $.
With $\delta(t) = [\delta_p(t)^T \ \delta_v(t)^T]^T$, $e_T = [e_f^T \ e_h^T]^T$, and $\Psi=\begin{bmatrix}
0 & 0\\
k_{\psi} \mathcal{M} & k_T \mathcal{M}
\end{bmatrix}$. 
The disagreement dynamics are given by
\begin{equation}
\dot{\delta}(t)=  \Sigma\delta(t)+\Psi e_T,
\end{equation}
and the solution is given by
\begin{equation*}
\delta(t)=e^{\Sigma t}\delta(0) +\int_0^t e^{\Sigma (t-\tau)}\Psi e_T d\tau.
\end{equation*} 
Then the disagreement vector is bounded by
\begin{equation*}
\| \delta(t) \| \leq  \| e^{\Sigma t} \delta(0)\| +  \int^{t}_{0} \|  e^{ \Sigma (t - \tau)} \Psi e_T\| d\tau.
\end{equation*}
Applying equation (\ref{eq:bounddi}) and using  $\kappa_2 = \kappa_2(\Sigma)$ yields
\begin{align*}
\|\delta(t)\| \leq & c_d e^{ -\kappa_2 t} \| \delta(0)\| \nonumber \\
    & +2  c_d \sqrt{N}(k_T+ ak_{\psi})  \int^{t}_{0} e^{ -\kappa_2 (t-\tau)} d\tau,
\end{align*}
because 
$\|\Psi e_T\|\leq k_{\psi}\|\mathcal{M}\|\|e_f\|+k_T\|\mathcal{M}\|\|e_h\|$, 
 $\|e_f\|\leq \|P(\psi)\hat{g}- P(\psi^*)\hat{g}^*\| \leq 2a\sqrt{N}$, 
 $\|e_h\|\leq \|\bar{T}\hat{g}- \bar{T}\hat{g}^*\| \leq 2 \sqrt{N}$,
 $\|\mathcal{M}\|=1$,  
 $\|\bar{T}\|=1$,
 $\|P(\psi)\|=a$,
$\| \tilde{g}_i \| \leq \mu$, 
and $\| \hat{g} \| \leq \sqrt{N}$.  
Then 
$\|\Psi e_T\| \leq 2 \sqrt{N} (k_T+ ak_{\psi})$.
Finally 
\begin{align}
\|\delta(t)\| \leq & \frac{2 c_d \sqrt{N}}{\kappa_2}(k_T+ak_{\psi}) \nonumber \\
& + \left( c_d\|\delta(0)\| - \frac{2 c_d\sqrt{N}}{\kappa_2} (k_T+ak_{\psi}) \right) e^{-\kappa_2 t}.
\end{align}
%This estimate shows that the zero-input response decays to zero exponentially fast while the zero-state response is bounded for every bounded input. 

Since $\boldsymbol{1}^T\mathcal{L}=\boldsymbol{0}^T$, the time derivative of $\bar{e}_p$ and $\bar{e}_v$ is given by 
\begin{align}\label{eq:lct_diaverage}
\dot{\bar{e}}_p(t) & = \bar{e}_v(t), \nonumber\\
\dot{\bar{e}}_v(t) & = -\gamma \bar{e}_v(t) + k_{\psi}\bar{e}_f(e_p)+ k_T\bar{e}_h(e_p),
\end{align}
with initial average $\bar{e}_p(0)=\frac{1}{N}\boldsymbol{1}^Te_p(0)=\bar{e}_{p0}$,  $\bar{e}_v(0)=\frac{1}{N}\boldsymbol{1}^Te_v(0)=\bar{e}_{v0}$,  
$\bar{e}_f=\frac{1}{N} \boldsymbol{1}^T \left(P(\psi) \hat{g}(e_p + r^*)-P(\psi^*)\hat{g}(r^*) \right)$, $\bar{e}_f$ satisfies $\bar{e}_f(0)=0$, and $\bar{e}_p^T \bar{e}_f(e_p) <0$ for all $e_p\neq 0$; % (Similar to a mass-spring-damper system). 
$\bar{e}_h=\frac{1}{N} \boldsymbol{1}^T \left( \bar{T}\hat{g}(e_p + r^*)-\bar{T}\hat{g}(r^*) \right)$, $\bar{e}_h$ satisfies $\bar{e}_h(0)=0$, and $\bar{e}_p^T \bar{e}_h(e_p) <0$ for all $e_p\neq 0$. % (Similar to a mass-spring-damper system).

Each agent's state converges to the consensus dynamics' solution of equation (\ref{eq:lct_diaverage}) if the graph is an undirected  and  consensus is achieved.  
Then equilibrium is stable because agents starting on any place of the scalar field  move towards the desired level curve.

To arrive to the same conclusion, consider the system
\begin{align*}
\dot{\bar{e}}_p(t)  = &\bar{e}_v(t), \\
\dot{\bar{e}}_v(t)  = &-\gamma \bar{e}_v(t) - s_f(e_p) - s_h(e_p),
\end{align*}
where  $s_f(e_p)=- k_{\psi}\bar{e}_f(e_p)$  and satisfies $s_f(0)=0$; $\bar{e}_p^T(e_p)s_f(e_p) > 0$ for all $e_p \neq 0$. 
$s_h(e_p)=- k_{\psi}\bar{e}_h(e_p)$  and satisfies $s_h(0)=0$; $\bar{e}_p^T(e_p)s_h(e_p) > 0$ for all $e_p \neq 0$.

Consider the Lyapunov function candidate with line integrals as
\begin{align}
 V=\frac{1}{2}\bar{e}_v^T\bar{e}_v + \int_0^{\bar{e}_p} s_f^T(z)dz + \int_0^{\bar{e}_p} s_h^T(z)dz,
\end{align}
where $V$ is continuously differentiable,  $V(0)=0$, and $V>0$ for all $e_p \neq0$ and $e_v \neq 0$. Then  
\begin{align}
\dot{V}=-\gamma \bar{e}_v^T\bar{e}_v \leq 0.
\end{align}
Applying the Lasalle's principle, it is possible to conclude that the equilibrium point is stable because  $\bar{e}_v=0 \Rightarrow \bar{e}_p=0\Rightarrow \dot{\bar{e}}_v=0=-s_f(0)=-s_h(0)=0$.  %Note that the feedback gains $k_T$ and $\gamma$ determines the consensus dynamics.  $k_F$ and $\theta$ determine both, whether the consensus is achieved and they also determine consensus velocity. 
\end{proof}


%\begin{remark}
%If  the estimated gradient is normalized, 
%$\hat{g}^N(t) = \left[ \frac{\hat{g}^T_1(t)}{\| \hat{g}_1(t) \|} ,\ldots, \frac{\hat{g}^T_N(t)}{\| \hat{g}_N(t) \|} \right]^T$,  $\| \frac{\hat{g}_i}{\| \hat{g}_i \| }\| = 1$ and $\mu =1$, then  $\| \hat{g}^N \| = \sqrt{N}$.
%Thus, the convergence region is given by
%\begin{align*}
%\|\delta(t)\| \leq & \frac{2 c_d \sqrt{N}\mu}{\kappa_2}(k_T+ak_{\psi}),
%\end{align*}
%and the agents travel at constant velocity.
%\end{remark}

%\begin{remark}
%The radius (\ref{eq:radiusdi}) can be reduced if  $k_T$ is reduced.  
%\end{remark}

%\begin{remark}
%The disagreement  dynamics are asymptotically stable for $e_g(t)=0$. Theorem 1 shows that $\|\delta(t)\|$ is bounded for bounded $\|e_g(t)\|$ and converge to zero asymptotically if $e_g(t)$ vanishes.
%\end{remark}

%\begin{remark}
%If  the source of the scalar field moves at constant velocity, $\|\dot{r}_s\|=V_s$,  and  the average velocity of the agents is greater than the velocity of the source, $\|\dot{\bar{r}}(t)\| > \|\dot{r}_s\|$, then the agents locate the source  and converge to the radius (\ref{eq:radiusdi}).
%\end{remark}








%%%%%%%%%%%%%%%%%%%%%%%%%%%%%%%%%%%%%%%%%%%%%%%%%%%%


\section{Cooperative Level Curve Tracking for LTI Models}  \label{sec:lc_ltimodel}

Consider a group of $N$ identical agents with linear time-varying dynamics described by
\begin{align}\label{eq:lct_LTImodel}
\dot{x}_i(t) & = A_px_i(t) + B_u u_i(t) + B_w \bar{w}_i(t),\nonumber \\
z_i(t) & =  C_z x_i (t) + D_{zu}u_i + D_{zw} \bar{w}_i(t), \nonumber \\
r_i(t) & =  C_y x_i (t) + D_{yw} \bar{w}_i(t),
\end{align}
where 
$x_i \in \mathbb{R}^n$ is the state of agent $i$, 
$u_i \in \mathbb{R}^{m_1}$ is the control input, 
$\bar{w}_i \in \mathbb{R}^{m_2}$ is the external disturbance and noise, 
$r_i \in \mathbb{R}^p$ is the measured position output, 
%$y_{vi} \in \mathbb{R}^p$ is the measured velocity output, 
and $z_i \in \mathbb{R}^q$ is the controlled output of agent $i$. 
It is assumed that ($A_p$, $B_u$) is stabilizable, ($A_p$, $C_y$) is detectable, and without loss of generality, $B_u$ is full column rank.

Based on the neighbours' relative output measurements between agents, a distributed dynamic output feedback controller  for agent $i$ is considered as
\begin{align}\label{eq:lct_LTIcontroller}
\dot{v}_i(t)  =& A_K v_i(t) +B_K (e_{Fi}(t) + e_{Ti}(t)),\nonumber\\ 
u_i(t) = & C_K v_i(t) +D_K(e_{Fi}(t) +e_{Ti}(t)), \nonumber \\
e_{Fi}(t) =& \frac{1}{\left| \mathcal{N}_i \right|}\sum_{j \in \mathcal{N}_i} a_{ij} \left[ (r_{Fi}(t)-r_{Fj}(t)) -(r_i(t)-r_j(t))  \right], \nonumber \\
e_{Ti}(t) =& v_{Ti}(t)=k_{\psi}(\psi_{ref}-\psi_i)\frac{\hat{g}_i}{\|\hat{g}_i\|}+ k_T T \frac{\hat{g}_i}{\|\hat{g}_i\|},
\end{align}
where
%$\hat{g}_i=\frac{\tilde{g}_i}{\|\tilde{g}_i\|}$, 
$r_{Fi}=Q(\alpha _i)\tilde{r}_{Fi}$, 
$
Q(\alpha_i)=
\begin{bmatrix}
\cos(\alpha_i) & \sin(\alpha_i)\\
\-sin(\alpha_i) &  \cos(\alpha_i)
\end{bmatrix},
$
$\alpha_i= \tan^{-1}\left(\frac{\hat{g}_{yi}}{\hat{g}_{xi}}\right)$ and 
$v_i(t) \in \mathbb{R}^{m_K}$ is the state of the dynamic output feedback controller.  %If $m_K=0$ the equation (\ref{eq:LTIcontroller}) is reduced to be a static output coordinating law.  
Let 
$x=[x_1^T, \ldots,  x_N^T]^T$, 
$v=[v_1^T, \ldots,  v_N^T]^T$, 
$\bar{w}=[\bar{w}_1^T, \ldots,  \bar{w}_N^T]^T$,
$r_F=[r_{F1}^T, \ldots,  x_{FN}^T]^T$, 
$v_T=[v_{T1}^T, \ldots,  v_{TN}^T]^T$, 
and 
$z=[z_1^T, \ldots, z_N^T]^T$. 

%As presented in \citep{RoseroWerner14c}, the source seeking problem is solved based on the distributed trajectory control law as 
%\begin{align}\label{eq:lti_ssp}
%e_{Ti}(t) =& v_{Ti}(t)=k_T \hat{g}_i(t).
%\end{align}
 

Substituting the controller (\ref{eq:lct_LTIcontroller}) into the system  (\ref{eq:lct_LTImodel}), the closed-loop network dynamics can be written as 
\begin{equation}\label{eq:lct_clLTIdynamic}
\begin{split}
\begin{bmatrix}
\dot{x}\\
\dot{v}
\end{bmatrix}
= &
\begin{bmatrix}
I_N \otimes A_p - \mathcal{L} \otimes B_uD_{K}C_y & I_N \otimes B_uC_K\\
- \mathcal{L} \otimes B_{K}C_y & I_N \otimes A_K
\end{bmatrix}
\begin{bmatrix}
x\\
v
\end{bmatrix} \\
& +
\begin{bmatrix}
\mathcal{L} \otimes B_uD_{K} & I_N \otimes B_uD_{K} & I_N\otimes B_w-\mathcal{L}\otimes B_uD_KD_{yw}\\
\mathcal{L} \otimes B_{K}  & I_N \otimes B_{K} & -\mathcal{L} \otimes B_KD_{yw}
\end{bmatrix}
\begin{bmatrix}
r_F \\
v_T\\
\bar{w}
\end{bmatrix}, \\
z= &
\begin{bmatrix}
I_N \otimes C_z - \mathcal{L}\otimes D_{zw}D_KC_y & I_N\otimes D_{zw}C_K
\end{bmatrix}
\begin{bmatrix}
x\\
v
\end{bmatrix} \\
& + 
\begin{bmatrix}
\mathcal{L}\otimes D_{zw}D_K & I_N\otimes D_K & I_N \otimes D_{zw} -\mathcal{L}\otimes D_{zw}D_KD_{yw}
\end{bmatrix}
\begin{bmatrix}
r_F\\
v_T\\
\bar{w}
\end{bmatrix},  \\
r = &
\begin{bmatrix}
I_N \otimes C_y & 0 
\end{bmatrix}
\begin{bmatrix}
x\\
v
\end{bmatrix} 
+
\begin{bmatrix}
0 & 0 & I_N \otimes D_{yw}  
\end{bmatrix}
\begin{bmatrix}
r_F\\
v_T\\
\bar{w}
\end{bmatrix},
\end{split}
\end{equation}
where $w(t)=[r_F^T \ v_T^T \ \bar{w}^T]^T$.
%
%\subsection{$H_{\infty}$ performance analysis}
%
%Let $0=\lambda_1 < \lambda_2 \leq, \ldots, \leq \lambda_N$ be the eigenvalues of the Laplacian matrix $\mathcal{L}$. Since $\mathcal{L}$ is symmetric, there exists a unitary matrix $U \in \mathbb{R}^{N \times N}$ such that $U^{-1}\mathcal{L}U= \Lambda = diag[\lambda_1,  \ldots, \lambda_N]$. 
%Let 
%$x= (U \otimes I_n)\tilde{x}$, 
%$v= (U \otimes I_{mK})\tilde{v}$, 
%$z= (U \otimes I_{q})\tilde{z}$,  
%$\bar{w}= (U \otimes I_{m2})\tilde{\bar{w}}$, 
%$v_T= (U \otimes I_p)\tilde{v}_T$, 
%%$\hat{g}= (U \otimes I_N)\tilde{\hat{g}}$, 
%$r_F= (U \otimes I_p)\tilde{r}_F$. 
%Then, equation (\ref{eq:clLTIdynamic}) can be rewritten as 
%\tiny
%\begin{align}\label{eq:clLTIdynamicLamb}
%\begin{bmatrix}
%\dot{\tilde{x}}\\
%\dot{\tilde{v}}
%\end{bmatrix}
%= &
%\begin{bmatrix}
%I_N \otimes A_p - \Lambda \otimes B_uD_{K}C_y & I_N \otimes B_uC_K\\
%- \Lambda \otimes B_{K}C_y & I_N \otimes A_K
%\end{bmatrix}
%\begin{bmatrix}
%\tilde{x}\\
%\tilde{v}
%\end{bmatrix} \nonumber \\
%& +
%\begin{bmatrix}
%\Lambda \otimes B_uD_{K} & I_N \otimes B_uD_{K} & I_N\otimes B_w-\Lambda\otimes B_uD_KD_{yw}\\
%\Lambda \otimes B_{K}  & I_N \otimes B_{K} & -\Lambda \otimes B_KD_{yw}
%\end{bmatrix}
%\begin{bmatrix}
%\tilde{r}_F \\
%\tilde{v}_T \\
%\tilde{\bar{w}}
%\end{bmatrix}, \nonumber \\
%\tilde{z}= &
%\begin{bmatrix}
%I_N \otimes C_z - \Lambda \otimes D_{zw}D_KC_y & I_N\otimes D_{zw}C_K
%\end{bmatrix}
%\begin{bmatrix}
%\tilde{x}\\
%\tilde{v}
%\end{bmatrix} \nonumber \\
%& + 
%\begin{bmatrix}
%\Lambda \otimes D_{zw}D_K & I_N\otimes D_K & I_N \otimes D_{zw} -\Lambda \otimes D_{zw}D_KD_{yw}
%\end{bmatrix}
%\begin{bmatrix}
%\tilde{r}_F\\
%\tilde{v}_T\\
%\tilde{\bar{w}}
%\end{bmatrix}, \nonumber \\
%\tilde{r}= &
%\begin{bmatrix}
%I_N \otimes C_y & 0 
%\end{bmatrix}
%\begin{bmatrix}
%\tilde{x}\\
%\tilde{v}
%\end{bmatrix} 
%+
%\begin{bmatrix}
%0 & 0 & I_N \otimes D_{yw}  
%\end{bmatrix}
%\begin{bmatrix}
%\tilde{r}_F\\
%\tilde{v}_T\\
%\tilde{\bar{w}}
%\end{bmatrix}. 
%\end{align}
%\normalsize
%Note  equation (\ref{eq:clLTIdynamicLamb}) is composed of $N$ individual systems as
%\small
%\begin{align}\label{eq:clLTIdynamicagent}
%\begin{bmatrix}
%\dot{\tilde{x}}\\
%\dot{\tilde{v}}
%\end{bmatrix}
%= &
%\begin{bmatrix}
% A_p - \lambda_i  B_uD_{K}C_y &  B_uC_K\\
%- \lambda_i  B_{K}C_y &  A_K
%\end{bmatrix}
%\begin{bmatrix}
%\tilde{x}\\
%\tilde{v}
%\end{bmatrix} \nonumber \\
%& +
%\begin{bmatrix}
%\lambda_i  B_uD_{K} &  B_uD_{K} &  B_w-\lambda_i  B_uD_KD_{yw}\\
%\lambda_i  B_{K}  &  B_{K} & -\lambda_i  B_KD_{yw}
%\end{bmatrix}
%\begin{bmatrix}
%\tilde{r}_F \\
%\tilde{v}_T \\
%\tilde{\bar{w}}
%\end{bmatrix}, \nonumber \\
%\tilde{z}= &
%\begin{bmatrix}
% C_z - \lambda_i  D_{zw}D_KC_y &  D_{zw}C_K
%\end{bmatrix}
%\begin{bmatrix}
%\tilde{x}\\
%\tilde{v}
%\end{bmatrix} \nonumber \\
%& + 
%\begin{bmatrix}
%\lambda_i  D_{zw}D_K &  D_K &  D_{zw} -\lambda_i  D_{zw}D_KD_{yw}
%\end{bmatrix}
%\begin{bmatrix}
%\tilde{r}_F\\
%\tilde{v}_T\\
%\tilde{\bar{w}}
%\end{bmatrix}, \nonumber \\
%\tilde{r}= &
%\begin{bmatrix}
% C_y & 0 
%\end{bmatrix}
%\begin{bmatrix}
%\tilde{x}\\
%\tilde{v}
%\end{bmatrix} 
%+
%\begin{bmatrix}
%0 & 0 &  D_{yw}  
%\end{bmatrix}
%\begin{bmatrix}
%\tilde{r}_F\\
%\tilde{v}_T\\
%\tilde{\bar{w}}
%\end{bmatrix},
%\end{align}
%\normalsize
% $\forall \ i=1, \ldots, N$. For $\lambda_1=0$, the state matrix of the system (\ref{eq:clLTIdynamic}) is unstable if the given matrix $A_p$ is unstable.
% 
%%\begin{align}
%%\begin{bmatrix}
%%\dot{\tilde{x}}\\
%%\dot{\tilde{v}}
%%\end{bmatrix}
%%= &
%%\begin{bmatrix}
%% A_p  &  B_uC_K\\
%%0 &  A_K
%%\end{bmatrix}
%%\begin{bmatrix}
%%\tilde{x}\\
%%\tilde{v}
%%\end{bmatrix} 
%% +
%%\begin{bmatrix}
%%0 &  B_uD_{K} &  B_w\\
%%0  &  B_{K} & 0
%%\end{bmatrix}
%%\begin{bmatrix}
%%\tilde{r}_F \\
%%\tilde{v}_T \\
%%\tilde{\bar{w}}
%%\end{bmatrix}, \nonumber \\
%%\tilde{z}= &
%%\begin{bmatrix}
%% C_z  &  D_{zw}C_K
%%\end{bmatrix}
%%\begin{bmatrix}
%%\tilde{x}\\
%%\tilde{v}
%%\end{bmatrix} 
%% + 
%%\begin{bmatrix}
%%0 &  D_K &  D_{zw} 
%%\end{bmatrix}
%%\begin{bmatrix}
%%\tilde{r}_F\\
%%\tilde{v}_T\\
%%\tilde{\bar{w}}
%%\end{bmatrix}, \nonumber \\
%%\tilde{r}= &
%%\begin{bmatrix}
%% C_y & 0 
%%\end{bmatrix}
%%\begin{bmatrix}
%%\tilde{x}\\
%%\tilde{v}
%%\end{bmatrix} 
%%+
%%\begin{bmatrix}
%%0 & 0 &  D_{yw}  
%%\end{bmatrix}
%%\begin{bmatrix}
%%\tilde{r}_F\\
%%\tilde{v}_T\\
%%\tilde{\bar{w}}
%%\end{bmatrix},
%%\end{align}
%%$\forall \ i=1, \ldots, N$. 
%
%Denote the transfer function matrices of systems (\ref{eq:clLTIdynamic}) and (\ref{eq:clLTIdynamicLamb})  by $T_{\tilde{w} \tilde{z}}$ and $T_{\tilde{w}_i \tilde{z}_i}$, respectively. Then it follows
%$
%T_{\tilde{w} \tilde{z}}  = \text{diag}(T_{\tilde{w}_1 \tilde{z}_1}, \ T_{\tilde{w}_2 \tilde{z}_2},\  \ldots, \ T_{\tilde{w}_N \tilde{z}_N}), %\nonumber \\
%%&
% = (U^{-1} \otimes I_{m2})T_{\tilde{w} \tilde{z}}(U \otimes I_{m1}).
%$
%Thus, the relationships between the $H_{\infty}$ norm of $T_{w z}$, $T_{\tilde{w} \tilde{z}}$ and  $T_{\tilde{w}_i \tilde{z}_i}$ is
%$
%\|T_{w z}\|_{\infty} = \|T_{\tilde{w} \tilde{z}}\|_{\infty}= \text{max}_{i=1,2,\ldots,N} \|T_{\tilde{w}_i \tilde{z}_i}\|_{\infty}.
%$
%%\begin{remark} 
% The previous equations convert the distributed $H_{\infty} $ control problem of the multi-agent network into a $H_{\infty} $ control problem of a single agent.
%%, thereby reducing the computational complexity significantly. The key tools leading to this result rely on the state, the input and the output transformation all together, is used in, e.g., \citep{MassioniVerhaegen09}, \citep{LiDuanChen11}.
%This transformations are based on papers \citep{LiDuanChen11}, \citep{MassioniVerhaegen09}.
%%\end{remark}
%
%%\begin{remark}
%% For the case $\lambda_1 = 0$, 
%% %the close-loop dynamic of equation (\ref{eq:clLTIdynamicLamb}) is the same  as a single agent in (\ref{eq:LTImodel}) with $u_i=0$. %Therefore the $H_{\infty}$ performance limit of system (\ref{eq:clLTIdynamic}) is not less than $H_{\infty}$ norm of (\ref{eq:LTImodel}) with $u_i=0$.
%%% (The $H_{\infty}$ norm of the whole agent network of (\ref{eq:clLTIdynamic}) is bigger than o equal to an isolated agent, implying  that the disturbance rejection level of the network coupled via (\ref{eq:LTIcontroller}) does not become any better, as compared to that of an isolated agent). 
%%the state matrix of the system (\ref{eq:clLTIdynamic}) is unstable if the given matrix $A_p$ is unstable.
%%% This implies that at least one state feedback will be needed in controller (\ref{eq:LTIcontroller}), i.e., graph $\mathcal{G}$ ca not be simple but must have at least one loop in order to reach a better $H_{\infty}$ performance.
%%
%%%For the formation control many controllers have been designed. The distributed controllers proposed in \citep{OlfatiMurray04}, \citep{BorrelliKeviczky08}, \citep{MassioniVerhaegen09},  \citep{PopovWerner09}, and \citep{PilzPopovWerner09}, state feedbacks are required in the controllers to all the agents. The distribute controller proposed in  \citep{LiDuanChen11} and \citep{LiDuanHuang09} need only a subset of agents to know their own states.
%%%The distributed controllers proposed in  \citep{LiuJiaDuYuan09}, \citep{LiuJia10} and \citep{XuXieLiLum13}, use an undetermined system matrix and a condition in terms of linear matrix inequalities (LMI).
%%\end{remark}


\subsection{Controller Synthesis}

%A closed-loop representation  of our approach  is shown in Fig. \ref{fig:LTIcontroller}. We consider each agent as locally controlled.
%In order to stabilize the formation of the agents, 
%First we design controllers to stabilize the formation, then we  analyze if the agents' formation locate the scalar field's source.

The proposed distributed controller has two parts: formation controller and tracking controller. Formation controller maintains agents in a desired relative position and tracking controller steers  agents in the gradient's perpendicular direction in order to track a level curve.

Based on the previous works, \citep{PilzPopovWerner09} and \citep{PopovWerner09} a robust control approach to formation control is implemented. First, a local controller to stabilize a single quad-rotor helicopter is designed. Second, a robust formation controller to maintain agents in a desired relative position is designed.

%\begin{figure}[ht]
%  \centering
%    \includegraphics[width=3.4in]{LTIcontroller.eps}
%  \caption{Closed-loop representation of a formation}
%  \label{fig:LTIcontroller}
%\end{figure}

To design a local controller $K_L$, a full state feedback LQR controller is used as  $\zeta_i=K_Lx_i +u_i$, where $K_L$ stabilizes local dynamics of each agent $i$ and $u_i$ is the control law for the formation and tracking part. $u_i$ steers agent's position in the formation and tracks the gradient direction. The new dynamics after applying a local controller are given by
\begin{align}\label{eq:lct_LTImodelstable}
\dot{x}_i(t) & = (A_p + BK_L)x_i(t) + B_u u_i(t) + B_w \bar{w}_i(t),\nonumber \\
\dot{x}_i(t) & = Ax_i(t) + B_u u_i(t) + B_w \bar{w}_i(t),\nonumber \\
z_i(t) & =  C_z x_i (t) + D_{zu}u_i + D_{zw} \bar{w}_i(t), \nonumber \\
r_i(t) & =  C_y x_i (t) + D_{yw} \bar{w}_i(t),
\end{align}
where $A=A_p+BK_L$ and the matrix is Hurwitz.
 %(If we assume that the local feedback controller loop is unstable, then, since zero is always an eigenvalue of $\mathcal{L}$ one of the systes in (ref) will be unstable and unobservable, hence no formation-level controller can stabilize it. If the local-feedback loop is stable, so are the systems in (ref) and one can select $K_F=0$, thus decoupling the systems while preserving stability).
%Theorem 3 and 4 of \citep{PilzPopovWerner09} is used to design the formation controller.  %A controller $K_F$ stabilizes the closed-loop formation in Fig. 3 for any number of agents and any fixed as well as time-varying communication topology  if $\| T(s) \|_{\infty}< 1$, where $T(s)=\textsf{lft}(G(s), K(s))$as shown in Fig. 4. 
The formation controller $K_{FT}$ is designed based on Theorems 3 and 4 of \citep{PilzPopovWerner09}. Performance requirements are incorporated into our design by means of  mixed-sensitivity loop shaping.  $H_{\infty}$ synthesis technique is used to design the controller.  A generalized plant construction with sensitivity $W_S(s)$ and control sensitivity $W_K(s)$ is shown in Fig. \ref{fig:generalizedplant}, where $w_p=r_{F}$.
These theorems reduce the formation stability problem to an $H_{\infty}$ design problem for a single agent with uncertainty. They also guarantee stability for any formation  if controller $K_{FT}$ satisfies the design requirements.


%\begin{theorem} (Theorem 3 of paper \citep{PopovWerner09}). 
%Consider the closed-loop formation in Fig.3. A controller $K(s)$ as in (2) stabilized the formation under every time-varying topology if it satisfies $\|T(s)\|_{\infty}<1$.
%\end{theorem}
%
%\begin{theorem} (Theorem 4 of paper \citep{PopovWerner09}). 
%Consider the closed-loop formation in Fig.3. Let $\Delta := \{ \delta_{\lambda}I_q \| |\delta_{\lambda}| \leq 1 \}$. A controller $K(s)$ as in (2) stabilized the formation under every fixed topology if is satisfies $\mu_{\Delta}(T(s)) < 1$.
%\end{theorem}


\begin{figure}[!ht]
  \centering
    \includegraphics[width=4.5in]{generalizedplant.eps}
  \caption{Generalized plant.}
  \label{fig:generalizedplant}
\end{figure}


%Next, the mixed-sensitivity performance requirements is incorporated into the design.  The generalized plant $\bar{G}(s)$  augmented with exogenous inputs $w_p$ and performance outputs $z_p$ is shown in figure. The construction of such a generalized plant with sensitivity $W_S(s)$ and control-sensitivity $W_K(s)$ filters is shown in figure ... where $w_p=r$.

%Note that the sensitivity filter in the above generalized plant... imposes a penalty on the relative error $e$ of the quad-ropter in the information. i.e.,  $\tilde{e}=\lambda(\tilde{r}-\tilde{y})=\tilde{r} -\tilde{y} +\delta_{\lambda}(\tilde{r}- \tilde{y})$ 

%-----





%
%The state space realization of the closed loop-system of equation (\ref{eq:clLTIdynamicagent}) is given by
%\begin{align}
%\dot{x}_{ci}(t) = & A_{ci}x_i(t) +B_{ci}w_i(t)\nonumber \\
%z_i(t) =& C_{ci} x_i(t)+D_cw_i(t)
%\end{align}
%with $x_i(0)=0$, and 
%
%$
%A_{ci}=
%\begin{bmatrix}
%A  - \lambda_i B_uD_{KF}C_p & B_uC_K\\
% - \lambda_i B_{KF}C_p  &  A_K
%\end{bmatrix}
%$, 
%$
%B_{ci} = 
%\begin{bmatrix}
%\lambda_i B_uD_{KF} &  B_uD_{KT}\\
%\lambda_i B_{KF}  &  B_{KT}
%\end{bmatrix}
%$, 
%$
%C_c = 
%\begin{bmatrix}
% C_p & 0 \\
%C_v & 0
%\end{bmatrix}
%$, 
%
%$
%D_c=
%\begin{bmatrix}
% I_p & 0 \\
%0 &  I_p
%\end{bmatrix}
%$.
%
%The transfer function from $w$ to $z$ of the closed-loop system is defined as
%\begin{align}
%T_{\tilde{w}_i \tilde{z}_i}(s)=
%\begin{bmatrix}
%A_{ci} & B_{ci}\\
%C_{ci} & D_c
%\end{bmatrix}
%=
%C_c(sI-A_{ci})^{-1}B_{ci} + D_c
%\end{align}
%
%The matrices $A_{ci}$, $B_{ci}$, $C_{c}$ and  $D_c$ depend on the distributed controller. The problem now is to design a controller such that constraints on the eigenvalue locations fo $A_{ci}$ and $\|T_{\tilde{w}_i \tilde{z}_i}\|_{\infty}$ are satisfied.
% 
%The $H_{\infty}$ norm for each agent $i$ is
%\begin{equation}
%\|T_{\tilde{w}_i \tilde{z}_i}\|_{\infty}^2=max_{w \neq 0}\frac{\int_0^{\infty}\tilde{z}_i^T\tilde{z}_i dt}{\int_0^{\infty}\tilde{w}_i^T\tilde{w}_i dt}
%\end{equation}
%After some computations, and using the variable transformation $Y_{Fi}=K_FC_pP_i$ and $Y_{Ti}=K_TC_vP_i$ leads to the following result:
%a necessary and sufficient condition  to achieve a $\|T\|_{\infty} < \gamma$ is the existence of matrices $P=P^T \succ 0$ and $Y_{Fi}$ and $Y_{Ti}$ that satisfy
%





%%%%%%%%%%%%%%%%%%%%%%%%%%%%%%%%%%%%%%%%%%%%%%%%%%%%%%%%%%%%%%%%%%%%%%%%%%%%%%%%%%


\subsection{Stability Analysis}  \label{sec:lc_stability}
Convergence of agents inside a scalar field is analysed as follows. 


\begin{theorem}
Consider the multi-agent system (\ref{eq:lct_LTImodel}) with control law (\ref{eq:lct_LTIcontroller}). Suppose that the assumptions \ref{as:1}, \ref{as:2}, \ref{as:4} and \ref{as:clc} are fulfilled. Then, for all $r_i(0) \in \mathbb{R}^p $ and $t \geq 0$,  agents track a level curve of an unknown scalar field $\psi(r)$ and the  closed-loop system converge to an equilibrium point centred ball with radius
\begin{align}\label{eq:lct_radiusLTI}
\epsilon \leq \frac{2\sqrt{N} c_{\hat{A}} (k_{\psi}a+k_T) \|C_y\| \left(\|\tilde{B}\| + \|\mathcal{L}\|\|\tilde{G}\|\right)}{\lambda_{\hat{A}}}.
\end{align}
\end{theorem}



\begin{proof}
Consider the closed-loop system of  (\ref{eq:lct_clLTIdynamic}), with $\bar{w}=0$, as
\begin{align}\label{eq:lct_clLTIdynamicw0}
\begin{bmatrix}
\dot{x}\\
\dot{v}
\end{bmatrix}
= &
\begin{bmatrix}
I_N \otimes A - \mathcal{L} \otimes B_uD_{K}C_y & I_N \otimes B_uC_K\\
- \mathcal{L} \otimes B_{K}C_y & I_N \otimes A_K
\end{bmatrix}
\begin{bmatrix}
x\\
v
\end{bmatrix} \nonumber \\
& +
\begin{bmatrix}
\mathcal{L} \otimes B_uD_{K} & I_N \otimes B_uD_{K} \\
\mathcal{L} \otimes B_{K}  & I_N \otimes B_{K} 
\end{bmatrix}
\begin{bmatrix}
r_F \\
v_T
\end{bmatrix}, \nonumber \\
%z= &
%\begin{bmatrix}
%I_N \otimes C_z - \mathcal{L}\otimes D_{zw}D_KC_y & I_N\otimes D_{zw}C_K
%\end{bmatrix}
%\begin{bmatrix}
%x\\
%v
%\end{bmatrix} \nonumber \\
%& + 
%\begin{bmatrix}
%\mathcal{L}\otimes D_{zw}D_K & I_N\otimes D_K 
%\end{bmatrix}
%\begin{bmatrix}
%r_F\\
%v_T
%\end{bmatrix}, \nonumber \\
r = & 
\begin{bmatrix}
I_N \otimes C_y & 0 
\end{bmatrix}
\begin{bmatrix}
x\\
v
\end{bmatrix}. 
\end{align}
\normalsize
%where $w(t)=[r_F^T \ v_T^T \ \bar{w}^T]^T$.


Let $\zeta=  [x^T \ v^T]^T$ and  $\xi = [r_F^T \ v_T^T]^T$. 
Equation (\ref{eq:lct_clLTIdynamicw0}) can be written as
\begin{align}
\dot{\zeta}= & \hat{A}\zeta + \hat{B}\xi,  \nonumber \\
r= &  \hat{C} \zeta,             
\end{align}
where $\hat{A}=I_N \otimes \tilde{A} + \mathcal{L} \otimes \tilde{F}$ and this term is Hurwitz matrix,  $\hat{B}=I_N \otimes \tilde{B} +  \mathcal{L} \otimes \tilde{G}$, $\hat{C}=I_N \otimes \tilde{C}$, and 
 
\begin{align*}
\tilde{A}=&
\begin{bmatrix}
A & B_uC_K \\
0 & A_K
\end{bmatrix}, \
\tilde{F} = 
\begin{bmatrix}
-B_uD_KC_y & 0\\
B_K & 0
\end{bmatrix}, \\
\tilde{B}= &
\begin{bmatrix}
0 & B_uD_K \\
0 & B_K
\end{bmatrix}, \
\tilde{G} = 
\begin{bmatrix}
-B_uD_K & 0\\
B_K & 0
\end{bmatrix}, \\
\tilde{C}= &
\begin{bmatrix}
C_y & 0
\end{bmatrix}.
\end{align*}
The equilibrium point is given by
\begin{align}
\zeta^*=&\hat{A}^{-1}\hat{B}\xi^*, \nonumber \\
r^*=&\hat{C}\zeta^*.
\end{align}

Let the state error $e_{\zeta}=\zeta - \zeta^*$ and the position error $e_r=r-r^*$.
The dynamic state error can be written as
\begin{align}
\dot{e} _{\zeta}=\hat{A}e_{\zeta}+\hat{B}(\xi- \xi^*),
\end{align}
where 
\begin{align*}
\xi- \xi^* = &
\begin{bmatrix}
0 \\
k_{\psi} \left( P(\psi)\hat{g}- P(\psi^*)\hat{g}^*\right) + k_T \bar{T}\left( \hat{g}-\hat{g}^*\right)
\end{bmatrix}, 
\end{align*} 
$\hat{g}=\hat{g}(e_r+r^*)$ and $\hat{g}^*=\hat{g}^*(r^*)$. 
The solution is given by
\begin{align}
e_{\zeta}=e^{\hat{A}t}e_{\zeta}(0)+\int_0^t e^{\hat{A}(t-\tau)}\hat{B}( \xi-\xi^* )d\tau,
\end{align}
and the position error can be written as
\begin{align}
e_r & =\hat{C} e_{\zeta}, \nonumber \\
      & =\hat{C} e^{\hat{A}t}e_{\zeta}(0)+\hat{C}\int_0^t e^{\hat{A}(t-\tau)}\hat{B} ( \xi-\xi^* )d\tau.
\end{align}
The solution is bounded by
\small
\begin{align}
\|e_r\|  \leq  & c_{\hat{A}}\|\hat{C}\| \|e_{\zeta}(0)\| e^{-\lambda_{\hat{A}} t} + \nonumber \\
& 2\sqrt{N}\mu (k_{\psi}a+k_T) c_{\hat{A}} \|\hat{C}\| \|\hat{B}\| \int_0^t e^{-\lambda_{\hat{A}}(t-\tau)}d\tau,
\end{align}
\normalsize
because $\|e^{\hat{A}t}\| \leq c_{\hat{A}}e^{-\lambda_{\hat{A}}t}$ and $\| \xi-\xi^* \| \leq 2\sqrt{N}(k_{\psi}a+k_T)$, since $\|P(\psi)\| \leq a$, $\|\bar{T}\| =1$, $\|\hat{g}\| \leq \sqrt{N}$ . 
It follows
\begin{align}
\|e_r\| \leq & \frac{2\sqrt{N} c_{\hat{A}} (k_{\psi}a+k_T) \|\hat{C}\|\|\hat{B}\|}{\lambda_{\hat{A}}} +\nonumber \\
 & c_{\hat{A}}\|\hat{C}\|\left( \|e_{\zeta}(0)\|- \frac{2\sqrt{N} (k_{\psi}a+k_T)\|\hat{B}\|}{\lambda_{\hat{A}}} \right)e^{-\lambda_{\hat{A}}t}.
\end{align}
Finally
\footnotesize
\begin{align}
\|e_r\| \leq & \frac{2\sqrt{N} c_{\hat{A}} (k_{\psi}a+k_T) \|C_y\| \left(\|\tilde{B}\| + \|\mathcal{L}\|\|\tilde{G}\|\right)}{\lambda_{\hat{A}}} \nonumber\\ 
&+ c_{\hat{A}}\|C_y\|\left( \|e_{\zeta}(0)\| \right)e^{-\lambda_{\hat{A}}t} - \nonumber \\
 & c_{\hat{A}}\|C_y\|\left( \frac{2\sqrt{N} (k_{\psi}a+k_T) \left(\|\tilde{B}\| + \|\mathcal{L}\|\|\tilde{G}\|\right)}{\lambda_{\hat{A}}} \right)e^{-\lambda_{\hat{A}}t},
\end{align}
\normalsize
since $\|\hat{C}\| \leq \|C_y\|$, $\|\hat{B}\| \leq \|\tilde{B}\| + \|\mathcal{L}\|\|\tilde{G}\|$.
\end{proof}


%
%\begin{remark}
%The radius (\ref{eq:radiusLTI}) can be reduced if  ... is increased and/or  $k_T$ is reduced.  
%\end{remark}


%\begin{remark}
%The disagreement  dynamics are asymptotically stable for $e_g(t)=0$. Theorem 1 shows that $\|\delta(t)\|$ is bounded for bounded $\|e_g(t)\|$ and converges to zero asymptotically if $e_g(t)$ vanishes.
%\end{remark}

%\
%
%\begin{remark}
%If  the estimated gradient is normalized, 
%$\hat{g}^N(t) = \left[ \frac{\hat{g}^T_1(t)}{\| \hat{g}_1(t) \|},\ldots, \frac{\hat{g}^T_N(t)}{\| \hat{g}_N(t) \|} \right]^T$, then $\| \frac{\hat{g}_i}{\| \hat{g}_i \| }\| = 1$, $\mu=1 $ and  $\| \hat{g}^{N} \| = \sqrt{N}$.
%Thus, the convergence region is defined by
%\begin{align}
%\|e_r\| \leq & \frac{2\sqrt{N} c_{\hat{A}} (k_{\psi}a+k_T) \|C_y\| \left(\|\tilde{B}\| + \|\mathcal{L}\|\|\tilde{G}\|\right)}{\lambda_{\hat{A}}}.
%\end{align}
%%and the average velocity of agents are given by $\|\dot{\bar{r}}(t)\| \leq k_T$. ?? %Note that agents maintain the formation and it does not mater where the agents are. 
%and the agents travel at constant velocity.
%\end{remark}


%\begin{remark}
%If  the source of the scalar field moves at constant velocity, $\|\dot{r}_s\|=V_s$,  and  the average velocity of the agents is bigger than the velocity of the source, $\|\dot{\bar{r}}(t)\| > \|\dot{r}_s\|$, i.e., $k_T>V_s$, then  the agents locate the source  and converge to the radius (\ref{eq:radiusLTI}).
%\end{remark}




%%%%%%%%%%%%%%%%%%%%%%%%%%%%%%%%%%%%%%%%%%%%%%%%%%%%%%%%

\section{Simulation Results}  \label{sec:lc_simul}

To evaluate the proposed level curve tracking algorithms, a formation of $7$ mobile agents with a communication graph $\mathcal{G}$ as shown in Fig. \ref{fig:basic_topology} is considered. The communication topology is fixed and undirected.
%we provide simulation results showing  convergence of the proposed source seeking algorithms only for double integrator models.  
%This framework is used both in this paper (part 1) and in the second part \citep{RoseroWerner14c}. 
%The scenario simulated here is adopted from \citep{RoseroWerner14b} and \citep{RoseroWerner14c}. 
The scalar field is defined as
\begin{align*}
\psi(r)=A_0e^{-\left((r-r_s)^TH_1(r-r_s)\right)}+A_0e^{-\left((r-r_s)^TH_2(r-r_s)\right)},
\end{align*}
where $A_0=3$,
$H_1=
\begin{bmatrix}
\frac{1}{2\sigma_{x1}^2} & 0\\
0 & \frac{1}{2\sigma_{y1}^2}
\end{bmatrix}
$,  
$H_2=
\begin{bmatrix}
\frac{1}{2\sigma_{x2}^2} & 0\\
0 & \frac{1}{2\sigma_{y2}^2}
\end{bmatrix}
$, 
$\sigma_{x1}=30$, $\sigma_{y1}=75$, $\sigma_{x2}=80$ and $\sigma_{y2}=25$. %Their maximum is located at  $r_s=[40 \ 80]^T$.  
Fig. \ref{fig:levelcontour} shows the level curve described by before equations.

To estimate the gradient, equation (\ref{eq:estgradient}) should be implemented in absence of noise; when signals are corrupted by noise  equation (\ref{eq:noisegradest}) should be implemented. Initial conditions of the consensus filter are set as $\xi(0)=\nu(0)=0$. Tuning parameters are set to $\beta_{\xi}=1$ and $\beta_{\nu}=1.2$.  For $r_i$ and $\psi_i$, the noise covariance matrix is set to diag$(0.23\ 0.24\ 0.21\ 0.22 \ 0.23\ 0.24 \ 0.25)$ and diag$(0.23\ 0.26\ 0.24\ 0.26 \ 0.28\ 0.29\ 0.3)$, respectively. 

%\begin{figure}[ht]
%  \centering
%    \includegraphics[width=1in]{basic_topology.eps}
%  \caption{Formation and communication topology of agents.}
%  \label{fig:basic_topology}
%\end{figure}

In this scenario the goal is to track a desired level curve maintaining a desired geometric agent formation. At the same time, objective of the controller is to reject disturbance at high frequency noise.


\subsection{Double Integrator Models}


The desired formation $r_{Fi}$ for double integrator models is defined as $r_{F1}=[0 \ 0]^T$,  $r_{F2}=[-3 \ 6]^T$, $r_{F3}=[-6 \ 0]^T$,  $r_{F4}=[-3 \ -6]^T$, $r_{F5}=[3 \ -6]^T$,  $r_{F6}=[6 \ 0]^T$, and $r_{F7}=[3 \ 6]^T$.
Their initial positions  $r_{0i}$ are set to $r_{01}=[0 \ 0]^T$,  $r_{02}=[0 \ 1]^T$, $r_{03}=[1 \ 1]^T$,  $r_{04}=[1 \ 0]^T$, $r_{05}=[2 \ 0]^T$,  $r_{06}=[2 \ 1]^T$, and $r_{07}=[3 \ 0]^T$ with respect to any position inside of the scalar field.
%Due to lack of space, only simulation results  for double integrator models are presented. 
The distributed controller's tuning parameters are set to $k_F=1$, $\theta= 1$ and $\gamma= 1.5$. %The second eigenvalue $\kappa_2$ of $\Sigma$ is $\gamma=1.5$ and the convergence radius is defined as $\epsilon=0.00$. 
%The average velocity of agents are given by $\frac{k_T}{\gamma}=0.00$.  

\begin{figure}[ht]
  \centering
    \psfragfig[scale=1]{./figures/di_lct_perturb2}
     \caption{Level curve tracking using the algorithm (\ref{eq:lct_di_dcl}) %Tracking of different level curves + perturbation: Fig rx vs ry and velocity
  }
  \label{fig:di_lct_perturb}
\end{figure}

\begin{figure}[!ht]
  \centering
    \psfragfig[scale=1]{./figures/di_lct_change_ref_switch2}
  \caption{Level curve tracking combining the algorithms (\ref{eq:lct_di_ssp}) and  (\ref{eq:lct_di_dcl}) }
  \label{fig:di_lct_change_ref_switch}
\end{figure}

%\begin{figure}[ht!]
%       \centering
%       \psfragfig[scale=1]{./figures/levelcontour2}
%        \caption{Contour of the scalar field}
%  		\label{fig:levelcontour}
%\end{figure}

Fig. \ref{fig:di_lct_perturb} shows agents' velocity and the formation's response  when agents are inside of a scalar field. Due to the distributed level curve tracking controller (\ref{eq:lct_di_dcl}), agents start moving from their initial positions towards the desired level curve  $\psi_{ref}=1$ maintaining the desired formation. Note that  agents' velocity is higher when agents are far from the desired level curve because error $\psi_{ref}-\psi_i$ is large. When agents are close to the desired level curve, they move at a constant average velocity defined by $\frac{k_T}{\gamma}=0.4 \ m/s$. At time $t=600 \ s$  disturbance steps in agents $1,\ 3$ and $4$ have been  included. The controller rejects such disturbance steps and agents go back to the desired level curve and formation. When agents move along a strong arc, inner agents move slower and outer agents move faster than the average velocity of the formation. Their velocity converges to a common value and agents converge to the desired relative positions.

To avoid higher velocities when agents are far from the desired level curve and to drive agents at constant velocity to any place of the scalar field, two distributed controllers can be used. When agents are far, the source seeking algorithm (\ref{eq:lct_di_ssp}) can be implemented to enable agents to track the gradient direction. When agents are close to the desired level curve, the level curve tracking algorithm (\ref{eq:lct_di_dcl}) can be used to drive agents in a direction perpendicular to the gradient. Fig. \ref{fig:di_lct_change_ref_switch} illustrates the effect of combining these two controllers. When agents start moving, the source seeking algorithm is used and agents travel at an average velocity predefined as $\frac{k_T}{\gamma}=0.66 \ m/s$. At time $t=150 \ s$ the level curve tracking algorithm (\ref{eq:lct_di_dcl}) is used and agents travel at an average velocity predefined as $\frac{k_T}{\gamma}=0.5 \ m/s$. Note that the whole formation is able to track different level curves.  In this simulation the level curve reference is changing between $\psi_{ref}=1$ and $\psi_{ref}=2$. 

%(incluir analisis de velocidad, rotacion de los agentes, velocidad promedio constante, perturbacion, efecto del ruido, etc..)
%
%Tracking of a level curve for a moving scalar field (Fig. (a)), and expansion and contraction moving scalar field (Fig. (b)).. Include a video in internet page for the reviewers
 
 
\subsection{Quad-rotor Helicopter}

Simulation results for LTI models are presented in this section. 
%A formation network of $N=7$ identical mobile agents under undirected and connected communication topology $\mathcal{G}$   (Fig. \ref{fig:basic_topology}) is considered. 
Each agent is an underactuated and unstable multi-input/multi-output (MIMO) $12^{th}$ order dynamic model as proposed in \citep{LaraSanchezLozanoCastillo06} and \citep{PilzPopovWerner09},  with 4 inputs and 3 position outputs (quad-roptor helicopter). 
%The initial positions and the final position in the $r_{z}$-coordinate are considered zero for all agents. 
The desired formation $r_{Fi}$ and the initial positions of quad-rotor helicopters $r_{0i}$ are defined in the same way as for double integrator models with component $r_{zi}=0$ for all agents.

%The desired formation $r_{Fi}$ is defined as $r_{F1}=[0 \ 0 \ 0]^T$,  $r_{F2}=[-3 \ 6 \ 0]^T$, $r_{F3}=[-6 \ 0 \ 0]^T$,  $r_{F4}=[-3 \ -6 \ 0]^T$, $r_{F5}=[3 \ -6 \ 0]^T$,  $r_{F6}=[6 \ 0 \ 0]^T$, and $r_{F7}=[3 \ 6 \ 0]^T$.
%The initial positions of agents $r_{0i}$ are given by $r_{01}=[0 \ 0 \ 0]^T$,  $r_{02}=[0 \ 1 \ 0]^T$, $r_{03}=[1 \ 1 \ 0]^T$,  $r_{04}=[1 \ 0 \ 0]^T$, $r_{05}=[2 \ 0 \ 0]^T$,  $r_{06}=[2 \ 1 \ 0]^T$, and $r_{07}=[3 \ 0 \ 0]^T$.  

%For the purpose of estimating a gradient in each agent, equation  %(\ref{eq:lsmin}) or equation 
%(\ref{eq:noiseweightedgradest}) is implemented when the signal measurements are corrupted by zero-mean Gaussian noise. We set the noise covariance matrix to diag$(0.21\ 0.22\ 0.23\ 0.24 \ 0.25\ 0.26 \ 0.27)$ and diag$(0.26\ 0.27\ 0.26\ 0.27 \ 0.26\ 0.26 \ 0.27)$ for $r_i$ and $\psi_i$, respectively. Their initial conditions are $\xi(0)=\nu(0)=0$. The consensus filters' tuning parameters are set to $\beta_{\xi}=1.2$ and $\beta_{\nu}=1$. 

To stabilize a single agent a full state feedback LQR controller is chosen. The weighting matrices are

$R=diag(100, \ 0.1, \ 25, \ 25)$ and 

$Q=diag(0.04, \ 1,\ 0.04,\ 1,\ 0.5,\ 20,\ 0.25,\ 1,\ 10^3,\ 50,\ 10^3,\ 50)$.

To track the desired level curve and maintain agents in a desired relative position, the level curve tracking algorithm (\ref{eq:lct_LTIcontroller})  is implemented. The sensitivity and control sensitivity weighting filters are $W_s=I_3 \otimes \left(\frac{1}{s+0.0001} \right)$ and $W_K=I_4 \otimes \left(50 \frac{s+10^3}{s+10^6} \right)$. The $H_{\infty}$ synthesis technique is used to design the controller. The designed $H_{\infty}$ controller is of 19th order. The robust stability $H_{\infty}$ norm is $0.992<1$.

%Similar to the double integrator models, Fig.
%% (\ref{fig:LTI_lct_perturb}) and
%(\ref{fig:LTI_lct_change_ref_switch})
%demonstrate that the agents can track the level curves maintaining the desired formation.
In the case of a quad-rotor helicopter formation, results are similar to them obtained under a double integrator model scenario. 


%\begin{figure}[ht]
%  \centering
%    \includegraphics[width=3.8in]{di_lct_perturb.eps}
%    %\includegraphics[width=2.3in]{levelcontour.eps}
%  \caption{Level curve tracking for quad-roptor helicopter.}
%  \label{fig:LTI_lct_perturb}
%\end{figure}
%
%\begin{figure}[ht]
%  \centering
%    \includegraphics[width=3.8in]{di_lct_change_ref_switch.eps}
%    %\includegraphics[width=2.3in]{levelcontour.eps}
%  \caption{Level curve tracking for quad-roptor helicopter.}
%  \label{fig:LTI_lct_change_ref_switch}
%\end{figure}

\begin{figure}[!ht]
  \centering
    \includegraphics[width=5.5in]{qcopter_lct_mov_scalar_field.eps}
    %\includegraphics[width=2.3in]{levelcontour.eps}
  \caption{Level curve tracking for quad-rotor helicopters operating into a moving scalar field. Algorithm  \ref{eq:lct_LTIcontroller} has been used.}
  \label{fig:qcopter_lct_mov_scalar_field}
\end{figure}


Implementing the approach here presented,  agents drive to travel at constant velocity to any place into the scalar field. In real scenarios scalar fields may travel in different directions and also expand or contract; for example under influence of changing environmental conditions. If agents' velocity is less than the scalar field source's velocity and/or extension-contraction velocity of the scalar field, agents localize and track the desired level curve.  Fig. \ref{fig:qcopter_lct_mov_scalar_field} shows agents tracking the desired level curve $\psi_{ref}=1$ while the scalar field's source is moving along the line $r_s(t)= r_{s0}+0.15t$.  Here the concentration and position signals are corrupted by noise.  The source moves at velocity $\dot{r}_s=0.2 \ m/s$ while agents move at $\dot{\bar{r}}=0.5 \ m/s$. The blue line indicates the movement of the scalar field's source.

%Movies showing the simulation flight of the quad-roptor helicopter and the double integrator models inside of the extension-contraction and moving scalar field are available at the website: http://www.tuhh.de/~rtser/. 

Agents are able to track the desired level curve $\psi_{ref}=1$, while the scalar field's source is moving along the line $r_s(t)= r_{s0}+0.15t$ and the scalar field is expanding and contracting according to the function $A_0(t)=3 +sin(0.05t)$. The agents' velocity is $0.5 \ m/s$ and the velocity of the scalar field's source is $0.2 \ m/s$.

%\begin{figure}[ht]
%  \centering
%    \includegraphics[width=3.6in]{qcopter_lct_ext_cont_scalar_field.eps}
%    %\includegraphics[width=2.3in]{levelcontour.eps}
%  \caption{Level curve tracking for quad-roptor helicopter with a extension-contraction and moving scalar field.}
%  \label{fig:qcopter_lct_ext_cont_scalar_field}
%\end{figure}

%%%%%%%%%%%%%%%%%%%%%%%%%%%%%%%%%%%%%%%%%%%%%%%%%%%%%%%%%%%%%%%

\section{Conclusions} \label{sec:lc_concl}
In this chapter, the problem of cooperative level curve tracking has been discussed. Distributed controllers based on both an estimated gradient and a formation controller for double integrator models and LTI systems have been proposed. Agents are able to track a desired level curve. Theoretical analysis demonstrates that the algorithms here proposed enable agents  to converge to a discretionary level curve into the a scalar field while  geometric formation is preserved. When  concentration and position signals are corrupted by noise, distributed consensus filters are used to estimate the gradient and agents still also able to track a desired level curve.




%%%%%%%%%%%%%%%%%%%%%%%%%%%%%%%%%%%%%%%%%%%%%%%%%%%%%%%%%%%%%%%%%%%%%%%%%%%%%%
%
%\section{CONCLUSIONS}
%
%
%\subsection{Figures and Tables}
%
%
%
%   \begin{figure}[thpb]
%      \centering
%      \framebox{\parbox{3in}{We suggest that you use a text box to insert a graphic (which is ideally a 300 dpi TIFF or EPS file, with all fonts embedded) because, in an document, this method is somewhat more stable than directly inserting a picture.
%}}
%      %\includegraphics[scale=1.0]{figurefile}
%      \caption{Inductance of oscillation winding on amorphous
%       magnetic core versus DC bias magnetic field}
%      \label{figurelabel}
%   \end{figure}
%   
%   
%A conclusion section is not required. Although a conclusion may review the main points of the paper, do not replicate the abstract as the conclusion. A conclusion might elaborate on the importance of the work or suggest applications and extensions. 
%
%\addtolength{\textheight}{-12cm}   % This command serves to balance the column lengths
%                                  % on the last page of the document manually. It shortens
%                                  % the textheight of the last page by a suitable amount.
%                                  % This command does not take effect until the next page
%                                  % so it should come on the page before the last. Make
%                                  % sure that you do not shorten the textheight too much.
%
%
%
%%%%%%%%%%%%%%%%%%%%%%%%%%%%%%%%%%%%%%%%%%%%%%%%%%%%%%%%%%%%%%%%%%%%%%%%%%%%%%%%%
%\section*{APPENDIX}
%
%Appendixes should appear before the acknowledgment.
%
%
%
%
%
%
%
%
%\begin{figure}[h]
%  \centering
%    \includegraphics[width=3.4in]{di_ry_rx.eps}
%  \caption{Formation response for agents modelled as double integrators. }
%  \label{double_integrator_rx_ry_formation}
%\end{figure}
%\begin{figure}[h]
%  \centering
%    \includegraphics[width=3.4in]{di_rx_vs_time.eps}
%  \caption{$x$-positions of the formation for agents modelled as double integrators. }
%  \label{double_integrator_rx_t_formation}
%\end{figure}
%\begin{figure}[h]
%  \centering
%    \includegraphics[width=3.4in]{di_gx.eps}
%  \caption{Normalized estimated gradient $g_x$ for agents modelled as double integrators.}
%  \label{double_integrator_gx_t_formation}
%\end{figure}
%
%
%\begin{figure}[h]
%  \centering
%    \includegraphics[width=3.4in]{q_ry_rx.eps}
%  \caption{Formation response for agents modelled as double integrators. }
%  \label{double_integrator_rx_ry_formation}
%\end{figure}
%\begin{figure}[h]
%  \centering
%    \includegraphics[width=3.4in]{q_rx_vs_time.eps}
%  \caption{$x$-positions of the formation for agents modelled as double integrators. }
%  \label{double_integrator_rx_t_formation}
%\end{figure}






